\documentclass{article}
\usepackage{amsmath}
\usepackage{amssymb}
\usepackage{graphicx}
\usepackage[top=1 in, bottom=1 in, left=1 in, right=1 in]{geometry}
\newcommand{\unit}[1]{\ensuremath{\, \mathrm{#1}}}

\begin{document}

\begin{abstract}

\end{abstract}

\section{Introduction}

\section{Methodology}

Given the stringent financial constraints of energy consumers in the developing world, micro-grids must be sized according minimum acceptable capacity.
This implies understanding the temporal characteristics of high-priority consumer demand, and pairing micro-grid operational up-time with such characteristics.
This section of the article outlines a strategy for sizing micro-grid systems. 
Our strategy emphasizes visualizing, and understanding, the the seasonal and sub-daily temporal characteristics of prospective designs, such that the implications on high priority consumer demand may be understood. 
Within this section, several figures will be introduced in order to illustrate our general methodology. 
In section \ref{CaseStudy}, these figures will be reintroduced with an eye toward interpreting the contained data and making design decisions for a micro-grid near Segou, Mali. 

%% Introduce LEGP
We chose to express system reliability using using an energy based metric called Lack of Energy to Generate Probability, \emph{LEGP}, which was first introduced by Wissem et al. \cite{•}.
This metric is equal to the annual consumer demand which a micro-grid could not supply, divided by the annual consumer demand on the micro-grid.
We compute \emph{LEGP} using an energy balance model with an hourly time step. 
Within appendix 1, we provide our motivation for using \emph{LEGP}. 
With appendix 1, we also provide in depth explanation of our energy balance algorithm.

%% Introduce cost versus reliability curve

The first step in our design procedure is to create a cost versus reliability curve. 
Such curves allow the designer to understand the marginal cost of added micro-grid reliability.
An example of one such curve is illustrated as figure \ref{CVsLEGP}.
	\footnote{In figure \ref{CVsLEGP}, the cost for a reliability is the optimal 	combination of PV generation and battery storage which achieves that
	reliability.
Thus, PV generation and battery storage capacity do not have a fixed ratio. 
For more information on our optimization strategy refer to appendix 1.}
From figure \ref{CVsLEGP}, we observe that each incremental decrease in \emph{LEGP} becomes more expensive on a cost per \unit{kW\! \cdot \! hr} basis.
However, this curve does not help us determine which system reliability will be acceptable to the micro-grid's consumers.
To determine such, we must know the magnitude and temporal characteristics of energy shortfalls. 
Because the creation of a cost versus reliability curve relies on a high temporal frequency energy balance algorithm, this energy shortfall information is available. 
Adequately sizing a micro-grid becomes a matter of understanding the information on time of outages embedded in each point on the cost versus reliability curve. 

%% Cost versus reliability curve.

\begin{figure}[ht] 
  \centering
    \includegraphics[trim = 25mm 93mm 25mm 100mm, clip,width=.75\textwidth]{CvLEGP_hiRel.pdf}
  \caption{Cost in \unit{USD/kW\! \cdot \! hr} versus \emph{LEGP} of 
micro-grid with refrigerator base load.
The simulation uses the insolation profile from Segou, Mali.
\emph{LEGPs} range from 0.001 to 0.10}
\label{CVsLEGP}
\end{figure}

%% Introduce seasonal/monthly bar plot. 

As a first step toward understanding the temporal energy shortfall characteristics of a potential micro-grid design, we suggest plotting the \emph{LEGP} for each month. 
This allows micro-grid designers to observe seasonal variations in reliability and identify portions of the year that may be of concern.
They are then able to strategically target this areas with a finer level temporal resolution.
	\footnote{Seasonal variations in reliability may be the
	 result of variations in demand or solar-resource.
	Concern over a temporal region may result from a peak in \emph{LEGP}, or it 
	may also result from seasons which have high consumer demand priority.
	If it is found that seasonal reliability and demand priority are relatively 	
	constant, it is up to the discretion of the micro-grid designers to 
	conduct fine resolution temporal analysis on the entire year or to sample 	 
	certain months.}
An example of one such monthly \emph{LEGP} plot is illustrated as figure \ref{MonthBar}. 
This figure includes the monthly \emph{LEGPs} of three different micro-grid alternatives for a single location.
	
%% Monthly LEGP Plot

\begin{figure}[ht] 
  \centering
    \includegraphics[trim = 20mm 75mm 20mm 85mm, clip,width=.75\textwidth]{monthlyLEGPMali010305.pdf}
  \caption{A bar plot of monthly \emph{LEGP} given annual \emph{LEGPs} of 0.01, 0.03, and 0.05. 
  The underlying model relies on the weather profile of Segou, Mali.
  The demand data is for the micro-grid with a refrigerator base load. }
\label{MonthBar}
\end{figure}

%% Introduce Hourly Resolution Analysis

Once we have identified the seasonal areas of concern, we are then able to observe them on a sub-daily time frame. 
This allows the micro-grid designer to qualitatively and quantitatively understand micro-grid performance from a consumer perspective. 
For our sub-daily analysis we used hourly increments; however, larger, or smaller, increments can be used depending on data availability.
%
%% Hourly Reliability Bar Plot
%
One informative data analysis method is to plot collective time of day reliability for a month or season of interest. 
For example, using hourly increments, a 24 element data set would be created.
The first element would be the reliability from 12:00 AM to 1:00 AM during the multi-day period.
These plots allow micro-grid designers to observe long term reliability trends.
They also allow observe how micro-grid reliability corresponds to trends in peak, or high priority, electricity demand. 
An example of one such time-of-day reliability plot, for three different micro-grid scenarios, is illustrated as figure \ref{HourBar}.

%% Hourly LEGP Bar Plot 

\begin{figure}[ht] 
  \centering
    \includegraphics[trim = 20mm 75mm 20mm 78mm, clip,width=.85\textwidth]{timeOfSFByHourCountAug010305.pdf}
  \caption{Bar plot of hourly \emph{LEGP} during the month of August. 
  As indicated in the legend, the different bar column types are for micro-	
  grids with Annual \emph{LEGPs} of 0.01, 0.03, and 0.05.
  Superimposed on the bar plot is the average hourly demand in \unit{W\! \cdot \! hr}.
  The underlying model relies on the weather profile of Segou, Mali.
  The demand data is for the micro-grid with a refrigerator base load.}
\label{HourBar}
\end{figure}

%% Discuss LEG Maps
% I am really struggling with this paragraph. It is difficult to find the balance between generalness and specificity. Too general and the section becomes difficult to understad. Too specific, and I am discussing the material which was intended for another section.

Another method for visualizing energy shortfalls is to create a Lack of Energy to Generate, \emph{LEG}, map. 
\emph{LEG} is the amount of demand, in \unit{W\! \cdot \! hr}, that the micro-grid was unable to supply.
Examples of \emph{LEG} maps for sub-annual time periods are illustrated within figure \ref{LEGMaps}.
Each cell within the \emph{LEG} maps corresponds to one hour of micro-grid performance, and the color of the cell corresponds to the magnitude of energy shortfall. 
\emph{LEG} maps allow micro-grid designers to qualitatively analyze how energy shortfalls are distributed across days and weeks.
For example, they can determine if a low time of day reliability is the result of several small energy shortfalls, or a handful of complete blackouts. 
In addition, these figures, combined with an understanding of relevant weather and demand data, allow micro-grid designers to assess whether energy shortfalls are supply or demand driven. 
For example, energy shortfalls randomly spaced across days would suggest weather driven outages; whereas, energy shortfalls spaced at seven day intervals would suggest demand driven outages. 

%% LEG Maps figure 

\begin{figure}[ht] 
  \centering
    \includegraphics[trim = 20mm 37mm 20mm 40mm, clip,width=\textwidth]{LEGMap010305.pdf}
  \caption{Maps of Lack of Energy, \emph{LEG}, at an hourly resolution for the month of August.
   From left to right, subplots are for Annual \emph{LEGPs} of 0.01, 0.03, and 0.05. 
  The underlying model relies on the weather profile of Segou, Mali.
  The demand data is for the micro-grid with a refrigerator base load.}
\label{LEGMaps}
\end{figure}







\section{Case Study: Segou, Mali} \label{CaseStudy}

\section{Discussion}

\appendix	
\section{Appendix 1: \emph{LEGP} and Energy Balance Algorithm} \label{A1}

We chose this metric for two reasons.
First, our procedure allows use understand both the frequency and magnitude of energy shortfalls. 
Second, this metric is analogous to maximum annual capacity shortage which is a metric used by HOMER.


\section{Appendix 2}


\begin{thebibliography}{9}

\bibitem{cite1}

\end{thebibliography}
\end{document}