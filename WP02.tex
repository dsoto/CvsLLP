\documentclass{article}

\begin{document}
\newcommand{\unit}[1]{\ensuremath{\, \mathrm{#1}}}
\begin{abstract}
This publication presents a tool by which policy makers can
understand the trade-offs between increased capital cost and the
reliability of the system.
\end{abstract}

\section{Introduction}
%% basic introduction
Many of the communities lacking electricity today share two challenges,
distance from the existing electrical grid and lack of access to capital
to create distributed generation.
The remoteness of these locations requires the use of off-grid power
which is often more expensive per watt of generation.
Since these communities do not have abundant capital to invest in power
production and storage, it is important to reduce the investment cost
which reduces the delivered cost of electricity.
Using measured and simulated weather and demand data we calculate the
generation and storage costs at a given level of reliability.


%% wasteful guidelines
Traditional guidelines regarding system autonomy can lead to systems
that are more expensive than necessary.
In order to provide a system that has excellent autonomy despite
variations in solar resource is expensive.

Many off-grid situations may not have a thermal backup due to cost or
logistic difficulty that allows for production of electricity during a
solar shortfall.
We investigate the trade-offs between increased reliability and
increased expense of generation and storage.

%% previous research
Previous research by Markvart and Egido and Lorenzo construct curves
that specify the locus of system sizes that satisfy a given LLP.
Their work does not specify the lowest cost point on each of the
iso-reliability curves.
This work adds an estimation of the lowest cost point for a given
reliability and allows us to construct a marginal cost of reliability
curve.
We believe that this curve will be useful to planners of off-grid
electricity.

%% describe investigations that were carried out
We can further investigate the financial return that being able to sell
the power made available by increased capital costs to create increased
reliability.
This investigation will determine if there is an optimum operating point
from the operator perspective.
We also investigate whether the shape of this cost versus reliability
curve changes in response to different types of loads.

The last part of this work will be the investigation of these marginal
reliability costs given the types of loads and the per hour service
costs for providing these loads.

We also investigate the effect of weather patterns on the shape of the
cost vs. loss of energy to generate curve.


\section{Model}
The primary metric we use to evaluate a micro-grid's performance is called Lack of Energy to Generate Probability (LEGP) which was introduced by Wissem et al.\cite{}.
LEGP is equal to the demand that a microgrid was unable to meet divided by the total demand over a specified time frame. 
For our model the time frame was 8760 hours, or one year. 
We express system reliability using an energy metric, instead of time based metrics such as Loss of Load Probability (LOLP), because of the highly variable demand profiles of offgrid communities. 
Within many of the off-grid communities observed a vast majority of electricity demand occurred during the few hours following sunset.
Electricity was used for applications such as lighting, television, and radio. 
LOLP does not accurately describe the quality of a microgrid's service when there are several hours per day when the microgrid is operational but there is little to no electricity demand. 
LOLP also fails to account for periods, such as during the late evening and early morning, when the microgrid is not operational but there is no electricity demand.

The model we created contains two basic levels.
% perhaps I can put in a little schematic which illustrates the different levels and their interaction
The inner level conducts an energy balance analysis on an hourly timescale.
This analysis simulates the production, storage, distribution, and consumption of energy in a village microgrid for each hour over the course of a year. 
The energy balance level is called upon by an script which computes all of the PV generation and battery storage combinations which satisfy a particular LEGP. 
This script then determines the generation and storage combination with the minimum initial capital investment cost. 
The optimization is then executed for several LEGPs, resulting in a cost vs. reliability plot.

\subsection{Energy Balance Algorithm}

The energy balance model requires two sets of time series data. 
The first is a vector of normal to sun insolation in \unit{W/m^2} for each hour of the year.
The second time series data set is of hourly electricity demand in \unit{W\cdot hr} throughout the year. 
Given the geographic coordinates of the microgrid, the dates of the insolation measurements, and the ground reflectance of the location, we calculate the insolation on the collector, $I_C$, using the methodology outlined by Masters \cite{}. 
To convert available insolation into PV electricity generation, the nominal PV capacity, $P_{nom}$, must be specified by the model operator or an external script.
We estimate that the nominal PV capacity will only be reached at the maximum insolation value within the one year data set.

As seen in equation \ref{eq:A}, we then use a scaling factor, $ A_{pv}$ with units of \unit{m^2} to estimate the energy generated at all other times. Then as equation \ref{eq:EPV} illustrates, the generated energy, $E_{pv}$, is linearly scaled at all times, $t$, according to available insolation. 


\begin{equation} \label{eq:A}
A_{pv} = \frac{max(I_C)}{P_{nom}}
\end{equation}

\begin{equation} \label{eq:EPV}
E_{pv}(t) = A_{pv} \cdot I_C(t)
\end{equation}


and the maximum and minimum allowable battery charge levels in watt hours, must be specified.


\begin{equation}
E_B (t+1) = E_B (t) + E_{pv} (t) - E_{dem} (t)
\end{equation}


\begin{equation}
LEG(t) = E_{dem} (t) - (E_{pv} (t)+E_B(t-1)-E_{Bmin})
\end{equation}


\subsection{Optimization Algorithm}



\begin{equation}
LEGP = \frac{\sum_{t=1}^T LEG(t)}{\sum_{t=1}^T E_{dem} (t)}
\end{equation}





We calculate the minimum cost system for a given level of reliability.
We then plot the cost versus reliability of the system.
We note that the magnitude of the slope of the curve increases at higher
levels of reliability.


\section{Results}
\subsection{Cost vs LEGP Curve}
\subsection{Weather Patterns}


\end{document}


Outline:
--------
change in reliability vs cost curve based on types of load (light,
freeze, computer)
efficiency of loads and the effect on avoided generation and storage and
customer avoided tariff?

Questions and Themes
--------------------
How does consumption and payment variability affect price?
