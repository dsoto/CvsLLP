\documentclass{article}

\begin{document}

\begin{abstract}
This publication presents a tool by which policy makers can
understand the trade-offs between increased capital cost and the
reliability of the system.
\end{abstract}

\section{Introduction}
%% basic introduction
Many of the communities lacking electricity today share two challenges,
distance from the existing electrical grid and lack of access to capital
to create distributed generation.
The remoteness of these locations requires the use of off-grid power
which is often more expensive per watt of generation.
Since these communities do not have abundant capital to invest in power
production and storage, it is important to reduce the investment cost
which reduces the delivered cost of electricity.
Using measured and simulated weather and demand data we calculate the
generation and storage costs at a given level of reliability.


%% wasteful guidelines
Traditional guidelines regarding system autonomy can lead to systems
that are more expensive than necessary.
In order to provide a system that has excellent autonomy despite
variations in solar resource is expensive.

Many off-grid situations may not have a thermal backup due to cost or
logistic difficulty that allows for production of electricity during a
solar shortfall.
We investigate the trade-offs between increased reliability and
increased expense of generation and storage.

%% previous research
Previous research by Markvart and Egido and Lorenzo construct curves
that specify the locus of system sizes that satisfy a given LLP.
Their work does not specify the lowest cost point on each of the
iso-reliability curves.
This work adds an estimation of the lowest cost point for a given
reliability and allows us to construct a marginal cost of reliability
curve.
We believe that this curve will be useful to planners of off-grid
electricity.

%% describe investigations that were carried out
We can further investigate the financial return that being able to sell
the power made available by increased capital costs to create increased
reliability.
This investigation will determine if there is an optimum operating point
from the operator perspective.
We also investigate whether the shape of this cost versus reliability
curve changes in response to different types of loads.

The last part of this work will be the investigation of these marginal
reliability costs given the types of loads and the per hour service
costs for providing these loads.

We also investigate the effect of weather patterns on the shape of the
cost vs. loss of energy to generate curve.


\section{Model}
We use a energy balance model to simulate the production, storage,
distribution, and consumption of energy in a village microgrid.

The primary metric we used to evaluate system performance was Lack of Energy to Generate Probability (LEGP) which was introduced by Wissem et al. 

If the consumption in the village is greater than the production of the
generation and storage, we define a loss of energy to generate event
following the work of \cite{}.

We calculate the minimum cost system for a given level of reliability.
We then plot the cost versus reliability of the system.
We note that the magnitude of the slope of the curve increases at higher
levels of reliability.


\section{Results}
\subsection{Cost vs LEGP Curve}
\subsection{Weather Patterns}


\end{document}


Outline:
--------
change in reliability vs cost curve based on types of load (light,
freeze, computer)
efficiency of loads and the effect on avoided generation and storage and
customer avoided tariff?

Questions and Themes
--------------------
How does consumption and payment variability affect price?
