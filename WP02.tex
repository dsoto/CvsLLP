\documentclass{article}






\begin{document}

\begin{abstract}
This publication presents a tool by which policy makers can
understand the trade-offs between increased capital cost and the
reliability of the system.
\end{abstract}

\section{Introduction}
In order to provide a system that has excellent autonomy despite
variations in solar resource is expensive.
Many off-grid situations may not have a thermal backup due to cost or
logistic difficulty that allows for production of electricity during a
solar shortfall.

Previous research by Markvart and Egido and Lorenzo construct curves
that specify the locus of system sizes that satisfy a given LLP.
Their work does not specify the lowest cost point on each of the
isoreliability curves.
This work adds an estimation of the lowest cost point for a given
reliability and allows us to construct a marginal cost of reliability
curve.
We believe that this curve will be useful to planners of off-grid
electricity.

We can further investigate the financial return that being able to sell
the power made available by increased capital costs to create increased
reliability.
This investigation will determine if there is an optimum opeating point
from the operator perspective.
We also investigate whether the shape of this cost versus reliability
curve changes in response to different types of loads.

The last part of this work will be the investigation of these marginal 
reliability costs given the types of loads and the per hour service 
costs for providing these loads.

\section{Model}

\section{Results}

\section

\section{Other}



Outline:
--------

methodology for constructing reliability vs. cost curve (RVCC)

change in reliability vs cost curve based on types of load (light,
freeze, computer)

synthetic customer loads to create statistical study?

efficiency of loads and the effect on avoided generation and storage and
customer avoided tariff?




Figures:
--------







Questions and Themes
--------------------

What is the sensitivity in terms of price per kWh?

How does weather variability affect price?

How does consumption and payment variability affect price?



\end{document}

