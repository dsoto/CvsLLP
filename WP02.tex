\documentclass[11p]{article}
\usepackage{amsmath}
\usepackage{amssymb}
\usepackage{textcomp}
\usepackage{graphicx}
\usepackage[top=1 in, bottom=1 in, left=1 in, right=1 in]{geometry}
\newcommand{\unit}[1]{\ensuremath{\, \mathrm{#1}}}

\begin{document}

\begin{abstract}

\end{abstract}

\section{Introduction}

As the size of electricity distribution networks is decreased, so is the diversity of demand sources and the diversity of generation.
Thus, the smaller an electricity distribution grid, the more vulnerable it becomes to fluctuations in demand and electricity generation.
As a result, small standalone micro-grids must rely more heavily on energy storage in order to buffer variability and meet demand.  
The inclusion of storage capacity greatly increases the capital cost of micro-grids.
Thus, it is vital to appropriately size electricity storage capacity.

The issue of appropriately sizing small scale micro-grid installations is highly pertinent to the electrification of rural locations within the developing world. 
Given stringent financial limitations, consumers are unwilling and unable to pay for unnecessary surplus capacity. 
Moreover, cost versus reliability trade offs can be made, as long as, the micro-grid satisfies their basic energy requirements. \cite{Marawanyika} 
This article is focused specifically on the sizing of micro-grids with solar photovoltaic, PV, electricity generation and battery storage. 
Within the developing world, proper maintenance and repair of fossil fuel engine generators can be difficult. 
Likewise, gasoline and diesel supply chains can be expensive and unreliable.  
As a result, depending on the availability of other renewable resources, PV micro-grids may offer the least expensive and most reliable form of energy service. \cite{Nouni,WB}
PV infrastructure needs only minimal maintenance.
Maintenance consists primarily of lead-acid battery replacement which must be completed every two to five years. 
Another advantage of small scale-photovoltaic technology is its modularity.  
As long as additional hardware constraints are satisfied, PV generation and battery storage capacity can be increased to meet the growing needs of first time electricity customers. \cite{Wamukonya}  

Recognizing the importance of proper micro-grid sizing, numerous studies have been conducted with the goal of designing stand alone systems for a specified reliability. 
These studies often rely upon a single or multi-year time series of solar data in order to simulate micro-grid performance. 
References \cite{Markvart,Arun,Hadj} used time series simulations to develop curves of PV electricity generation versus battery storage for a desired reliability. 
Knowing the per unit cost of PV and battery installation, it is then possible to determine the lowest cost option on a generation versus storage iso-reliability curve. 
Hadj Arab et al. stresses the importance of conducting time series analysis and iso-reliability optimization. 
They illustrate that the shape of an iso-reliability curve is dependent upon a location's weather profile, the relative cost of PV and battery installations, and the desired micro-grid reliability. 
Moreover, they stress that these influences may be lost while using more simple methods. \cite{Hadj}   

Within the developed world, micro-grid consumers can afford and expect 100 percent system reliability.
Thus, the role of micro-grid designers is to choose the lowest cost PV and battery combination on an iso-reliability curve for 100 percent up-time.
However, consumers in the developing world are more willing to sacrifice reliability in favor of decreased cost.
In such a case, micro-grid designers should determine the minimum cost solution for a series of reliablities. 
The designers should then choose a design from the locus of minimum cost solutions which meets the consumers' basic reliability requirements and fits within the consumers' budget constraints. 
Researchers \cite{Hadj,Kanase,Wissem} have tangentially observed the cost versus reliability trade offs for stand alone micro-grid designs.  
The work of Kanase-Pital et al \cite{Kanase} observes how the cost of delivered energy, and the composition of hybrid renewable energy systems changes as a function of reliability.  
Similarly, Wissem et al \cite{Wissem} observe how the reliability of autonomous PV systems influences the cost per \unit{kW \! \cdot \! hr} of electricity. 
Hadj Arab et al. \cite{Hadj} determine the least expensive PV to battery ratio for several system reliabilities. 
Using their methodology, it is then possible to compute a cost versus reliability relationship for a prospective micro-grid project. 

Existing research illustrates the feasibility of calculating cost versus reliability relationships for potential micro-grid designs.
However, this research does not provide an overarching strategy to determine which cost-reliability combination is best for a particular group of consumers.  
To address the shortfall in existing literature, we propose a methodology by which a system planner may subjectively determine an optimum point where both cost and customer satisfaction are met.
Drawing on the work of others, we create a locus of optimal PV and battery combinations for a range of system reliabilities.
Embedded in each point on the curve is a one year temporal simulation of micro-grid performance with hourly resolution.  
We then iteratively observe points on the curve while observing when energy shortfalls occur and how they correlate to consumer demand patterns.
When analyzing a potential micro-grid design, we begin our temporal analysis by quantifying its reliability for each month of the year. 
This allows us to observe seasonal trends in micro-grid performance. 
After identifying the month or months of principal importance, or lowest reliability, we isolate those periods and analyze their performance using sub-daily resolution. 
Depending on he accuracy of the input parameters, the results should be quantitatively and qualitatively representative of trends in micro-grid performance. 
From observing these periods, system designers are then able to determine the acceptability of system performance.


Within section \ref{method}: methodology, we present a detailed description of our procedure for designing standalone PV micro-grids. 
This section introduces several figures.
These figures are used to demonstrate data visualization techniques which help to evaluate the performance of a potential micro-grid.  
Section \ref{CaseStudy} is an illustrative example of how we used our proposed micro-grid design strategy to design a micro-grid design for consumers in Segou, Mali. 
Within section \ref{CaseStudy}, the figures from section \ref{method} are reintroduced in order to illustrate how they informed our design decisions.
Using section \ref{Discussion}, discussion, we BLAH BLAH BLAH!!!!!!!!!!!!!


\section{Methodology} \label{method}

%drs can we add a more specific overview of how we will proceed?  
%we should lay out the seasonal->month->hourly clearly in the first
%paragraph.
%this info is good but might be more relevant to the introduction.  

%drs here is my attempt at the first paragraph for this section
Within this section, we describe our methodology of identifying the lowest cost system that uses real or forecast customer demand and weather data and price estimates for photovoltaic panels and storage.
In this method, we adjust the system reliability to minimize costs while simultaneously observing the times when the system is unable to meet demand.  
We begin our temporal analysis by quantifying the micro-grid reliability for each month of the year. 
This allows us to observe seasonal trends in micro-grid reliability. 
After identifying the month or months of principal importance, or lowest reliability, we isolate those periods and analyze their performance using sub-daily resolution. 
This iterative procedure is subjective but allows a system designer with knowledge of local demands to design a lower cost system without the complexity of backup fossil fuel based generation.

%% Introduce LEGP and Introduce cost versus reliability curve



\begin{verbatim}
** I think the next paragraph is important, I'm just not sure if it belongs here.**
\end{verbatim} 
The first step in our design procedure is to input insolation and demand data. 
Several strategies have been employed to estimate electricity demand on new micro-grid installations for the developing world. 
The authors of \cite{Camblong,Alzola} used extensive surveying to estimate the magnitude of electricity demand on a per load consuming device basis. 
The surveys also included time of day information about usage which employed to used create a daily power profile with hourly resolution. 
Nfah et al. \cite{Nfah} employed a different strategy.
They estimated the electricity demand of future installation by using historical data of grid connected households with similar electricity consumption behavior. 
Tools are also available which assist in estimating solar resource data.
Given the scarcity of ground-based solar resource data, these tools have been developed to convert geostationary imagery into site-time specific solar data. 
For instance, Mines ParisTech and the Center for Energy and Processes utilized Meteosat geostationary satellite imaging to create insolation data sets for all of Africa and Western Europe. 
For most locations, from 2005 to the present these data sets have a spacial resolution finer than 10 km and a temporal solution finer than 1 hour.  
Similarly, the NASA Langley Research Center has completed a worldwide solar energy data set using satellite imagery. 
The data set embodies 20 years of data with a 100-km spacial resolution and a daily temporal resolution.  
NASA has also used this data to create an average daily insolation for each month at each location.
As a result, this data may easily imported into a tool such as HOMER in order to create synthetic daily sets at sub-daily resolution.   

After input energy demand and insolation data has been specified, the next step in our design procedure is to create a cost versus reliability curve. 
Such curves allow the designer to understand the marginal cost of added micro-grid reliability.
An example of one such curve is illustrated as figure \ref{CVsLEGP}.
As indicated in figure \ref{CVsLEGP}, the metric we use in order to express system reliability is called Lack of Energy to Generate Probability, \emph{LEGP}. 
First introduced by Wissem et al. \cite{Wissem}, \emph{LEGP} is an energy based metric which is equal to the annual consumer demand a micro-grid could not supply, divided by the annual consumer demand on the micro-grid.
We compute \emph{LEGP} using an energy balance model with an hourly time step. 
Within appendix 1, we provide in depth explanation of our energy balance algorithm.
Within appendix 1, we also provide our motivation for using \emph{LEGP}.


From figure \ref{CVsLEGP}, we observe that each incremental decrease in \emph{LEGP} becomes more expensive on a cost per \unit{kW\! \cdot \! hr} basis.
However, this curve does not help us determine which system reliability will be acceptable to the micro-grid's consumers.
To determine such, we must know the magnitude and temporal characteristics of energy shortfalls. 
Because the creation of a cost versus reliability curve relies on a high temporal frequency energy balance algorithm, this energy shortfall information is available. 
Adequately sizing a micro-grid becomes a matter of understanding the information on time of outages embedded in each point on the cost versus reliability curve. 

%% Cost versus reliability curve.

\begin{figure}[ht] 
  \centering
    \includegraphics[trim = 25mm 93mm 25mm 100mm, clip,width=.75\textwidth]{CvLEGP_hiRel.pdf}
  \caption{Cost in \unit{USD/kW\! \cdot \! hr} versus \emph{LEGP} of 
micro-grid with refrigerator base load.
The simulation uses the insolation profile from Segou, Mali.
\emph{LEGPs} range from 0.001 to 0.10.
The cost for a reliability is the optimal combination of PV generation and battery storage which achieves that reliability.
Thus, PV generation and battery storage capacity do not have a fixed ratio. 
For more information on our optimization strategy refer to appendix 1.}
\label{CVsLEGP}
\end{figure}

%% Introduce seasonal/monthly bar plot. 

As a first step toward understanding the temporal energy shortfall characteristics of a potential micro-grid design, we suggest plotting the \emph{LEGP} for each month. 
This allows micro-grid designers to observe seasonal variations in reliability and identify portions of the year that may be of concern.
They are then able to strategically target this areas with a finer level temporal resolution.
Seasonal variations in reliability may be the result of variations in demand or solar-resource.
Concern over a temporal region may result from a peak in \emph{LEGP}, or it may also result from seasons which have high consumer demand priority.
If it is found that seasonal reliability and demand priority are relatively 	
constant, it is up to the discretion of the micro-grid designers to 
conduct fine resolution temporal analysis on the entire year or to sample certain months.
An example of one such monthly \emph{LEGP} plot is illustrated as figure \ref{MonthBar}. 
This figure includes the monthly \emph{LEGPs} of three different micro-grid alternatives for a single location.
	
%% Monthly LEGP Plot

%drs - keep in mind that plots serve as tools in the procedure.  be sure
%you describe them in the context of the procedure.

\begin{figure}[ht] 
  \centering
    \includegraphics[trim = 20mm 75mm 20mm 85mm, clip,width=.75\textwidth]{monthlyLEGPMali010305.pdf}
  \caption{A bar plot of monthly \emph{LEGP} given annual \emph{LEGPs} of 0.01, 0.03, and 0.05. 
  The underlying model relies on the weather profile of Segou, Mali.
  The demand data is for the micro-grid with a freezer base load. }
\label{MonthBar}
\end{figure}

%% Introduce Hourly Resolution Analysis

Once we have identified the seasonal areas of concern, we are then able to observe them on a sub-daily time frame. 
This allows the micro-grid designer to qualitatively and quantitatively understand micro-grid performance from a consumer perspective. 
For our sub-daily analysis we used hourly increments; however, larger, or smaller, increments can be used depending on data availability.
%
%% Hourly Reliability Bar Plot
%
We are able to observe informative trends in time of day reliability by plotting it for a month or season of interest. 
For example, using hourly increments, a 24 element data set would be created.
The first element would be the reliability from 12:00 AM to 1:00 AM during the multi-day period.
An example of one such time-of-day reliability plot, for three different micro-grid scenarios, is illustrated as figure \ref{HourBar}.
These plots allow micro-grid designers to observe how the probability of energy shortfall changes throughout an average day.
From these trends, micro-grid designers visualize how reliability corresponds to trends in peak, or high priority, electricity demand. 


%% Hourly LEGP Bar Plot 

\begin{figure}[ht] 
  \centering
    \includegraphics[trim = 20mm 75mm 20mm 78mm, clip,width=.85\textwidth]{timeOfSFByHourCountAug010305.pdf}
  \caption{Bar plot of hourly \emph{LEGP} during the month of August. 
  As indicated in the legend, the different bar column types are for micro-	
  grids with Annual \emph{LEGPs} of 0.01, 0.03, and 0.05.
  Superimposed on the bar plot is the average hourly demand in \unit{W\! \cdot \! hr}.
  The underlying model relies on the weather profile of Segou, Mali.
  The demand data is for the micro-grid with a freezer base load.}
\label{HourBar}
\end{figure}



We are able to qualitatively analyze how energy shortfalls are distributed across days and weeks by creating a Lack of Energy to Generate, \emph{LEG}, map. 
\emph{LEG} is the amount of demand, in \unit{W\! \cdot \! hr}, that the micro-grid was unable to supply.
Examples of \emph{LEG} maps for sub-annual time periods are illustrated within figure \ref{LEGMaps}.
Each cell within the \emph{LEG} maps corresponds to one hour of micro-grid performance, and the color of the cell corresponds to the magnitude of energy shortfall. 
Using \emph{LEG} maps to understand inter-day reliability can be used to answer questions such as, "is a low time of day reliability the result of several small energy shortfalls, or a handful of complete blackouts?" 
In addition, these figures, combined with an understanding of relevant weather and demand data, allow micro-grid designers to assess whether energy shortfalls are supply or demand driven. 
For example, energy shortfalls randomly spaced across days would suggest weather driven outages; whereas, energy shortfalls spaced at seven day intervals would suggest demand driven outages. 

%drs add paragraph that wraps up and summarizes procedure

The primary purpose of our micro-grid design strategy is to isolate periods when a micro-grid is under the most stress and determine if its performance is acceptable.
If micro-grid performance is acceptable during this time, it will be acceptable during the rest of year.
The primary exception being that seasonal variations in micro-grid usage can enforce time of year variations in acceptable system reliability. 
Regardless, our procedure allows micro-grid designers to analyze the performance of high priority demand months. 


%% LEG Maps figure 

\begin{figure}[ht] 
  \centering
    \includegraphics[trim = 20mm 37mm 20mm 40mm, clip,width=\textwidth]{LEGMap010305.pdf}
  \caption{Maps of Lack of Energy, \emph{LEG}, at an hourly resolution for the month of August.
   From left to right, subplots are for Annual \emph{LEGPs} of 0.01, 0.03, and 0.05. 
  The underlying model relies on the weather profile of Segou, Mali.
  The demand data is for the micro-grid with a freezer base load.}
\label{LEGMaps}
\end{figure}


\section{Case Study: Segou, Mali} \label{CaseStudy}

%% Introduce Section

This section is intended as an illustrative example.
The methodology introduced in the previous section is utilized in order to size a micro-grid for a village outside of Segou, Mali.
There is currently a micro-grid on the site. 
From consumer feedback, the micro-grid is deemed to have an unacceptable reliability.
According to our model, the micro-grid has a \emph{LEGP} of 0.054.
In order to facilitate the design process, and motivate specific micro-grid design decisions, the figures from section \ref{method} are reintroduced.

\subsection{Input Parameters}

%% Introduce Location and weather profile 
The following subsection describes the input parameters of our model.
A majority of the inputs were drawn from the operational data of the current micro-grid. 
There are three primary input parameters for the micro-grid design strategy. 
Two are hourly vectors for year. One vector is of available insolation in \unit{W/m^2}, and the other is of electricity demand in \unit{w\! \cdot \! hr}.
A third set of input parameters is the annual costs of PV and battery capacity. 
%
% Introduce Location
%
Segou, Mali is located at 13\textdegree 27' 0" N, 6.13\textdegree 16' 0" W. 
This location thus has little variation in seasonal clear sky solar resource availability; the hours of daylight for the summer and winter solstices are 12 hours, 55 minutes and 11 hours, 20 minutes, respectively.  
%%
%% Seasonal variations in the weather.
%%
Although the clear sky irradiance is relatively constant throughout the year, irradiance at ground level is not.
Segou's weather is characterized by a rainy season which lasts from June until September, and a dry season throughout the rest of the year. 

%%
%% Introduce demand profile 
%%
A majority of the electricity demand on the micro-grid is used to operate
a 250 liter freezer.
The freezer operates nearly 24 hours per day, and on average it draws 200 W. 
Ice bricks and frozen drinks produced by the freezer are sold to neighboring villages.
In addition to powering the freezer, the micro-grid serves approximately 20 households.
Each household is equipped with two 5 W LED light bulbs and a two plug outlet. 
The outlets are used primarily for cellphone charging.
A few households may also have larger electronics, such as a television or radio.
All households are individually metered, and are charged on a per watt basis.
Thus, residential electricity peak demand is significantly less than the maximum possible power draw. 
On average, residential electricity demand peaks during the early evening at 150 W.
The daily average of the demand data used in our analysis is illustrated as the dashed line in figure \ref{HourBar}.

When conducting our financial analysis, we estimated the annual cost of PV capacity to be 0.1762 \unit{USD/W}.
We also estimated the annual cost of battery capacity to be 0.0804 \unit{USD/ W  \! \cdot \! hr}. 
For a more detailed explanation of how the weather, demand, and economic parameters were generated, please refer to Appendix \ref{A2}.

\subsection{Case Study Procedure and Results}

% Introduce Sub-Section

Our design procedure for the micro-grid in Segou, Mali, required several iterations.
Each iteration generated a unique micro-grid configuration. 
The \emph{LEGP}, PV capacity, battery capacity, and cost per \unit{kW \! \cdot \! hr} of select iterations are printed in table \ref{ConfigSumm}.

% Design alternative summary table

\begin{table}

\centering
\begin{tabular}{|l|c|c|c|c|c|}
\hline
\emph{LEGP}  &  PV Capacity & Bat Capacity & Combine Cost\\
      & \unit{W}     & \unit{W \! \cdot \!hr} & of PV and Bat Bank\\
       & & &  \unit{USD/kW \! \cdot \!hr}\\
\hline
Current Grid & & & \\
\emph{LEGP} = 0.054 & 1400 & 17280 & 0.881 \\
\hline
0.05 & 2000 & 7800 & 0.524 \\
\hline
0.03 & 2600 & 8400 & 0.591 \\
\hline
0.01 & 2500 & 13000 & 0.763 \\
\hline
\end{tabular}
\caption[Caption for LOF]{Performance, size, and costs characteristics of current and potential micro-grid's for Segou, Mali.\footnotemark}
\label{ConfigSumm}
\end{table}

\footnotetext{Note that the \emph{LEGP} and the cost per \unit{kW\! \cdot \! hr} were estimated by our model and do not constitute measured reliability or cost data.}
% 0.05 LEGP Micro-Grid

%Paragraph 2

When sizing the micro-grid for the village outside of Segou, we started by specifying an annual \emph{LEGP} of 0.05.
Recognizing the severe financial limitations on micro-grid design, we wanted to start with a reliability that was marginally better than the 0.054 \emph{LEGP} of the current system.
After specifying an annual \emph{LEGP} of 0.05, we plotted the \emph{LEGP} for each operational month. 
Figure \ref{MonthBar} illustrates that, as expected from the weather patterns of Segou, the months of July through September had the lowest reliability.  
Because it had the lowest reliability, with an \emph{LEGP} of 0.136, we chose to analyze the month of August with an hourly resolution. 

%Paragraph 3
 
In order to qualitatively understand the temporal spacing of energy shortages, and how they affect electricity demand, we created figures \ref{HourBar} and \ref{LEGMaps}.
After analyzing the figures, we concluded that the 0.05 \emph{LEGP} micro-grid did not offer adequate reliability for either the freezer base load or the resident demand.
We found that micro-grid was insufficient for the residential consumers because a significant number of energy shortages occur during times of significant residential demand.
As illustrated by figure \ref{HourBar}, we can see that there is a significant level of residential electricity demand between 7:00 PM and 1:00 AM, with peak demand occurring between 9:00 PM and 10:00 PM.
Between 7:00 PM and 1:00 AM reliability steadily decreases, with \emph{LEGP} rising from 0.052 to 0.205. 
During the peak demand window, which occurs between 9:00 PM until 10:00 PM, the \emph{LEGP} of the micro-grid was 0.138.
The 0.05 \emph{LEG} map within figure \ref{LEGMaps} confirms that the system provides insufficient electricity service to the residential customers. 
In particular, we can see that the high \emph{LEGPs} were the result of regularly occurring energy shortages and not isolated outages.
Figure \ref{LEGMaps} indicates that there were five energy shortfall events which curtailed demand during the 7:00 PM to 1:00 AM period.
Four of these inhibit electricity consumption during the hour of peak demand, 9:00 PM to 10:00 PM.
We found that the micro-grid failed to provide sufficient service for the freezer system because there were several outages of long duration.
We estimate that any energy shortfalls lasting five hours or longer would significantly impact the production of ice and frozen drinks. 
As figure \ref{LEGMaps} illustrates, energy shortfalls lasting five or more hours occurred prior to, or during, eight separate work days.
 
% Now it is on to do a brief summary 
% of the LEGP = 0.03 microgrid and Full summary of why 
% the LEGP = 0.01 micro-grid is wonderful.

Recognizing that an \emph{LEGP} of 0.05 was insufficient to meet our consumers' demand, we iteratively increased system reliability and observed the temporal characteristics using the section \ref{method} methodology.
An intermediate design was a micro-grid with an annual \emph{LEGP} of 0.03. 
We found that the \emph{LEGP} micro-grid was not sufficient and that its inadequacies closely paralleled those of the 0.05 \emph{LEGP} micro-grid.
Like the 0.05 \emph{LEGP} micro-grid, this design would have a significant negative impact on residential electricity usage during August.
Recognizing that most residential demand occurs between 7:00 PM and 1:00 AM, figure \ref{LEGMaps} illustrates that there were four instances in which power outages would inhibit residential electricity supply. 
With respect to the freezer operators, we found that there would be five outages which have a duration of five hours or longer. 
Subjectively, we determined that five days of lost revenue concentrated within a one month period would be unacceptable to the freezer operators. 

As a result of our iterative design process, we decided upon a micro-grid with an annual \emph{LEGP} of 0.01. After isolating the lowest reliability month, and analyzing it with an hourly resolution, we decided that the micro-grid was acceptable to residential consumers. 
%
% footnote
%
\footnote{Observing figure \ref{MonthBar}, September has a slightly higher \emph{LEGP} than August for the 0.01 annual \emph{LEGP} micro-grid. 
We present the hourly performance of August to facilitate reader comparison of design iterations. 
Although not presented, we also analyzed September with an hourly resolution, and the results confirmed our findings.}
%
%
%
Figure \ref{LEGMaps} indicates that there was only a single energy shortage between 7:00 PM and 1:00 AM, representing a single event in which residential consumption was significantly impacted.
We also found that the 0.01 \emph{LEGP} micro-grid significantly improved freezer operation, especially when compared to the aforementioned alternatives. There were only two energy shortfall occurrences which were five ours or longer. 
Given the reduced frequency and duration of energy shortfalls, it is possible that their effects can be further reduced by demand side energy management.

\section{Discussion} \label{Discussion}

\appendix	
\section{Appendix 1: \emph{LEGP} and Energy Balance Algorithm} \label{A1}

We chose this metric for two reasons.
First, our procedure allows use understand both the frequency and magnitude of energy shortfalls. 
Second, this metric is analogous to maximum annual capacity shortage which is a metric used by HOMER.


\section{Appendix 2: Explaination of Case Study Input Parameters} \label{A2}

\subsection{Weather Data}
\subsection{Demand Data}
\subsection{Cost Parameters}

\begin{thebibliography}{9}

\bibitem{Marawanyika} Marawanyika G. The Zimbabwe UNDP-G.E.F solar project for rural household and community use in Zimbabwe. Renewable Energy 1997; 10(2–3): 157–162

\bibitem{Nouni} Nouni MR, Mullick SC, Kandpal TC. Providing electricity access to remote areas in India: Niche areas for decentralized electricity supply. Renewable Energy 2009; 34(2): 430-4.

\bibitem{WB} World Bank, Rural energy and development: improving energy supplies for 2 billion people. World Bank, Washington, DC (1996)

\bibitem{Oparaku}Oparaku OU. Photovoltaic systems for distributed power supply in Nigeria. Renewable Energy 2002; 25:31-40.

\bibitem{Wamukonya} Wamukonya N. 2007. Solar home system electrification as a viable technology option for Africa’s development. Energy Policy 35, 6–14.

\bibitem{Markvart} Markvart T, Fragaki A, Ross JN. PV system sizing using observed time series of solar radiation, Solar Energy 2006; 80(1):46-50.

\bibitem{Arun} Arun P, Banerjee R,Bandyopadhyay S, Sizing curve for design of isolated power systems, Energy for Sustainable Development 2007; 11(4) 21-8.

\bibitem{Hadj} Hadj Arab A, Ait Driss B, Amimeur R, Lorenzo E, Photovoltaic systems sizing for Algeria, Solar Energy 1995; 54(2): 99-104.

\bibitem{Kanase} Kanase-Patil AB, Saini RP,Sharma MP, Sizing of integrated renewable energy system based on load profiles and reliability index for the state of Uttarakhand in India, Renewable Energy 2011; 36(11): 2809-21.

\bibitem{Wissem} Wissem Z, Gueorgui K, Hédi K, Modeling and technical–economic optimization of an autonomous photovoltaic system, Energy 2012; 37(1): 263-272.

\bibitem{Camblong} Camblong H, Sarr J, Niang AT, Curea O, Alzola JA, Sylla EH, Santos M, Micro-grids project, Part 1: Analysis of rural electrification with high content of renewable energy sources in Senegal, Renewable Energy 2009; 34(10): 2141-2150.

\bibitem{Alzola} Alzola JA, Vechiu I, Camblong H, Santos M, Sall M, Sow G, Microgrids project, Part 2: Design of an electrification kit with high content of renewable energy sources in Senegal, Renewable Energy 2009; 34(10):  2151-2159. 

\bibitem{Nfah} E.M. Nfah, J.M. Ngundam, M. Vandenbergh, J. Schmid, Simulation of off-grid generation options for remote villages in Cameroon, Renewable Energy, Volume 33, Issue 5, May 2008, Pages 1064-1072.


\end{thebibliography}
\end{document}


