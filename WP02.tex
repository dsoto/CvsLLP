\documentclass{article}

\newcommand{\unit}[1]{\ensuremath{\, \mathrm{#1}}}

\begin{document}

\begin{abstract}
This publication presents a tool by which policy makers can
understand the trade-offs between increased capital cost and the
reliability of the system.
\end{abstract}

\section{Introduction}
%% basic introduction
Many of the communities lacking electricity today share two challenges,
distance from the existing electrical grid and lack of access to capital
to create distributed generation.
The remoteness of these locations requires the use of off-grid power
which is often more expensive per watt of generation.
Since these communities do not have abundant capital to invest in power
production and storage, it is important to reduce the investment cost
which reduces the delivered cost of electricity.
Using measured and simulated weather and demand data we calculate the
generation and storage costs at a given level of reliability.
%% wasteful guidelines
Traditional guidelines regarding system autonomy can lead to systems
that are more expensive than necessary.
In order to provide a system that has excellent autonomy despite
variations in solar resource is expensive.
We investigate the trade-offs between increased reliability and
increased expense of generation and storage.
These curves allow a policy-maker or system installer to make informed
decisions about the predicted uptime of the system and the resulting
cost of the system.
If customers may be willing to go without service for a certain time period
each year, if the tariff is significantly lowered.

%% previous research
Previous research by Markvart and Egido and Lorenzo construct curves
that specify the locus of system sizes that satisfy a given LLP.
Their work does not specify the lowest cost point on each of the
iso-reliability curves.
%% how we build on previous research
This work adds an estimation of the lowest cost point for a given
reliability and allows us to construct a marginal cost of reliability
curve.
The shape of this reliability curve depends on the local weather
patterns as well as the load patterns.
This work demonstrates the differences in the shape of the cost vs.
reliability curve for different climates.
This allows a more accurate view of cost and autonomy than is available
from average monthly insolation data.




\section{Model}
The primary metric we use to evaluate a micro-grid's performance is called Lack of Energy to Generate Probability (LEGP) which was introduced by Wissem et al.\cite{}.
LEGP is equal to the demand that a microgrid was unable to meet divided by the total demand over a specified time frame. 
For our model the time frame was 8760 hours, or one year. 
We express system reliability using an energy metric, instead of time based metrics such as Loss of Load Probability (LOLP), because of the highly variable demand profiles of offgrid communities. 
Within many of the off-grid communities observed a vast majority of electricity demand occurred during the few hours following sunset.
Electricity was used for applications such as lighting, television, and radio. 
LOLP does not accurately describe the quality of a microgrid's service when there are several hours per day when the microgrid is operational but there is little to no electricity demand. 
LOLP also fails to account for periods, such as during the late evening and early morning, when the microgrid is not operational but there is no electricity demand.

The model we created contains two basic levels.
% perhaps I can put in a little schematic which illustrates the different levels and their interaction
The inner level conducts an energy balance analysis on an hourly timescale.
This analysis simulates the production, storage, distribution, and consumption of energy in a village microgrid for each hour over the course of a year. 
The energy balance level is called upon by an script which computes all of the PV generation and battery storage combinations which satisfy a particular LEGP. 
This script then determines the generation and storage combination with the minimum initial capital investment cost. 
The optimization is then executed for several LEGPs, resulting in a cost vs. reliability plot.

\subsection{Energy Balance Algorithm}

The energy balance model requires two sets of time series data. 
This first is a vector of normal to sun insolation $\unit{(W/m^2)}$ for each hour of the year. The second time series data set is of hourly electricity demand $\unit{(W\cdot hr)}$ throughout the year. 
Given the geographic coordinates of the microgrid, the dates of the insolation measurements, and the ground reflectance of the location, we calculate the insolation on the collector $(I_C)$ using the methodology outlined by Masters \cite{}. To convert available insolation into PV electricity generation, $E_{gen}$, the nominal PV capacity must be specified. We estimated that the nominal PV capacity will only be reached at the maximum insolation value of the year long data set. 

\begin{equation}
A_{pv} = \frac{max(I_C)}{P_{nom}}
\end{equation}

The generated energy is linearly scaled at all times, $t$, according to available insolation. 

\begin{equation}
E_{pv}(t) = A_{pv} \cdot I_C(t)
\end{equation}


and the maximum and minimum allowable battery charge levels in watt hours, must be specified.


\begin{equation}
E_B (t+1) = E_B (t) + E_{pv} (t) - E_{dem} (t)
\end{equation}


\begin{equation}
LEG(t) = E_{dem} (t) - (E_{pv} (t)+E_B(t-1)-E_{Bmin})
\end{equation}


\subsection{Optimization Algorithm}



\begin{equation}
LEGP = \frac{\sum_{t=1}^T LEG(t)}{\sum_{t=1}^T E_{dem} (t)}
\end{equation}





We calculate the minimum cost system for a given level of reliability.
We then plot the cost versus reliability of the system.
We note that the magnitude of the slope of the curve increases at higher
levels of reliability.


\section{Results}
\subsection{Cost vs LEGP Curve}
\subsection{Weather Patterns}


\end{document}
