\documentclass{article}
\usepackage{amsmath}
\usepackage{amssymb}
\usepackage{textcomp}
\usepackage{graphicx}
\usepackage[top=1 in, bottom=1 in, left=1 in, right=1 in]{geometry}
\newcommand{\unit}[1]{\ensuremath{\, \mathrm{#1}}}

\begin{document}

\begin{abstract}

\end{abstract}

\section{Introduction}

\section{Methodology} \label{method}

Given the stringent financial constraints of energy consumers in the developing world, micro-grids must be sized according minimum acceptable capacity.
This implies understanding the temporal characteristics of high-priority consumer demand, and pairing micro-grid operational up-time with such characteristics.
This section of the article outlines a strategy for sizing micro-grid systems. 
Our strategy emphasizes visualizing, and understanding, the the seasonal and sub-daily temporal characteristics of prospective designs, such that the implications on high priority consumer demand may be understood. 
Within this section, several figures will be introduced in order to illustrate our general methodology. 
In section \ref{CaseStudy}, these figures will be reintroduced with an eye toward interpreting the contained data and making design decisions for a micro-grid near Segou, Mali. 

%% Introduce LEGP
We chose to express system reliability using using an energy based metric called Lack of Energy to Generate Probability, \emph{LEGP}, which was first introduced by Wissem et al. \cite{•}.
This metric is equal to the annual consumer demand which a micro-grid could not supply, divided by the annual consumer demand on the micro-grid.
We compute \emph{LEGP} using an energy balance model with an hourly time step. 
Within appendix 1, we provide our motivation for using \emph{LEGP}. 
With appendix 1, we also provide in depth explanation of our energy balance algorithm.

%% Introduce cost versus reliability curve

The first step in our design procedure is to create a cost versus reliability curve. 
Such curves allow the designer to understand the marginal cost of added micro-grid reliability.
An example of one such curve is illustrated as figure \ref{CVsLEGP}.
	\footnote{In figure \ref{CVsLEGP}, the cost for a reliability is the optimal 	combination of PV generation and battery storage which achieves that
	reliability.
Thus, PV generation and battery storage capacity do not have a fixed ratio. 
For more information on our optimization strategy refer to appendix 1.}
From figure \ref{CVsLEGP}, we observe that each incremental decrease in \emph{LEGP} becomes more expensive on a cost per \unit{kW\! \cdot \! hr} basis.
However, this curve does not help us determine which system reliability will be acceptable to the micro-grid's consumers.
To determine such, we must know the magnitude and temporal characteristics of energy shortfalls. 
Because the creation of a cost versus reliability curve relies on a high temporal frequency energy balance algorithm, this energy shortfall information is available. 
Adequately sizing a micro-grid becomes a matter of understanding the information on time of outages embedded in each point on the cost versus reliability curve. 

%% Cost versus reliability curve.

\begin{figure}[ht] 
  \centering
    \includegraphics[trim = 25mm 93mm 25mm 100mm, clip,width=.75\textwidth]{CvLEGP_hiRel.pdf}
  \caption{Cost in \unit{USD/kW\! \cdot \! hr} versus \emph{LEGP} of 
micro-grid with refrigerator base load.
The simulation uses the insolation profile from Segou, Mali.
\emph{LEGPs} range from 0.001 to 0.10}
\label{CVsLEGP}
\end{figure}

%% Introduce seasonal/monthly bar plot. 

As a first step toward understanding the temporal energy shortfall characteristics of a potential micro-grid design, we suggest plotting the \emph{LEGP} for each month. 
This allows micro-grid designers to observe seasonal variations in reliability and identify portions of the year that may be of concern.
They are then able to strategically target this areas with a finer level temporal resolution.
	\footnote{Seasonal variations in reliability may be the
	 result of variations in demand or solar-resource.
	Concern over a temporal region may result from a peak in \emph{LEGP}, or it 
	may also result from seasons which have high consumer demand priority.
	If it is found that seasonal reliability and demand priority are relatively 	
	constant, it is up to the discretion of the micro-grid designers to 
	conduct fine resolution temporal analysis on the entire year or to sample 	 
	certain months.}
An example of one such monthly \emph{LEGP} plot is illustrated as figure \ref{MonthBar}. 
This figure includes the monthly \emph{LEGPs} of three different micro-grid alternatives for a single location.
	
%% Monthly LEGP Plot

\begin{figure}[ht] 
  \centering
    \includegraphics[trim = 20mm 75mm 20mm 85mm, clip,width=.75\textwidth]{monthlyLEGPMali010305.pdf}
  \caption{A bar plot of monthly \emph{LEGP} given annual \emph{LEGPs} of 0.01, 0.03, and 0.05. 
  The underlying model relies on the weather profile of Segou, Mali.
  The demand data is for the micro-grid with a freezer base load. }
\label{MonthBar}
\end{figure}

%% Introduce Hourly Resolution Analysis

Once we have identified the seasonal areas of concern, we are then able to observe them on a sub-daily time frame. 
This allows the micro-grid designer to qualitatively and quantitatively understand micro-grid performance from a consumer perspective. 
For our sub-daily analysis we used hourly increments; however, larger, or smaller, increments can be used depending on data availability.
%
%% Hourly Reliability Bar Plot
%
One informative data analysis method is to plot collective time of day reliability for a month or season of interest. 
For example, using hourly increments, a 24 element data set would be created.
The first element would be the reliability from 12:00 AM to 1:00 AM during the multi-day period.
These plots allow micro-grid designers to observe long term reliability trends.
They also allow observe how micro-grid reliability corresponds to trends in peak, or high priority, electricity demand. 
An example of one such time-of-day reliability plot, for three different micro-grid scenarios, is illustrated as figure \ref{HourBar}.

%% Hourly LEGP Bar Plot 

\begin{figure}[ht] 
  \centering
    \includegraphics[trim = 20mm 75mm 20mm 78mm, clip,width=.85\textwidth]{timeOfSFByHourCountAug010305.pdf}
  \caption{Bar plot of hourly \emph{LEGP} during the month of August. 
  As indicated in the legend, the different bar column types are for micro-	
  grids with Annual \emph{LEGPs} of 0.01, 0.03, and 0.05.
  Superimposed on the bar plot is the average hourly demand in \unit{W\! \cdot \! hr}.
  The underlying model relies on the weather profile of Segou, Mali.
  The demand data is for the micro-grid with a freezer base load.}
\label{HourBar}
\end{figure}

%% Discuss LEG Maps
% I am really struggling with this paragraph. It is difficult to find the balance between generalness and specificity. Too general and the section becomes difficult to understad. Too specific, and I am discussing the material which was intended for another section.

Another method for visualizing energy shortfalls is to create a Lack of Energy to Generate, \emph{LEG}, map. 
\emph{LEG} is the amount of demand, in \unit{W\! \cdot \! hr}, that the micro-grid was unable to supply.
Examples of \emph{LEG} maps for sub-annual time periods are illustrated within figure \ref{LEGMaps}.
Each cell within the \emph{LEG} maps corresponds to one hour of micro-grid performance, and the color of the cell corresponds to the magnitude of energy shortfall. 
\emph{LEG} maps allow micro-grid designers to qualitatively analyze how energy shortfalls are distributed across days and weeks.
For example, they can determine if a low time of day reliability is the result of several small energy shortfalls, or a handful of complete blackouts. 
In addition, these figures, combined with an understanding of relevant weather and demand data, allow micro-grid designers to assess whether energy shortfalls are supply or demand driven. 
For example, energy shortfalls randomly spaced across days would suggest weather driven outages; whereas, energy shortfalls spaced at seven day intervals would suggest demand driven outages. 

%% LEG Maps figure 

\begin{figure}[ht] 
  \centering
    \includegraphics[trim = 20mm 37mm 20mm 40mm, clip,width=\textwidth]{LEGMap010305.pdf}
  \caption{Maps of Lack of Energy, \emph{LEG}, at an hourly resolution for the month of August.
   From left to right, subplots are for Annual \emph{LEGPs} of 0.01, 0.03, and 0.05. 
  The underlying model relies on the weather profile of Segou, Mali.
  The demand data is for the micro-grid with a freezer base load.}
\label{LEGMaps}
\end{figure}

\section{Case Study: Segou, Mali} \label{CaseStudy}

%% Introduce Section

This section is intended as an illustrative example.
The methodology introduced in the previous section is utilized in order to size a micro-grid for a village outside of Segou, Mali.
The figures from section \ref{method} are also reintroduced.
As opposed to illustrating general data visualization strategies, these figures will be used to motivate specific micro-grid design decisions.


\subsection{Input Parameters}

%% Introduce Location and weather profile 

There are three primary input parameters for the micro-grid design strategy. 
Two are hourly vectors for year. One vector is of available insolation in \unit{w/m^2}, and the other is of electricity demand in \unit{w\! \cdot \! hr}.
A third set of input parameters is the annual costs of PV and battery capacity. 
%
% Introduce Location
%
Segou, Mali is located at 13\textdegree 27' 0" N, 6.13\textdegree 16' 0" W. 
This location is between the equator and the Tropic of Cancer, and thus has little variation in seasonal clear sky solar resource availability. 
The hours of daylight for the summer and winter solstices are 12 hours, 55 minutes and 11 hours, 20 minutes, respectively.  
%%
%% Seasonal variations in the weather.
%%
Although the clear sky irradiance is relatively constant throughout the year, irradiance at ground level is not.
Segou's weather is characterized by a rainy season which lasts from June until September, and a dry season throughout the rest of the year. 

%%
%% Introduce demand profile 
%%
A majority of the electricity demand on the micro-grid is used to operate
a 250 liter freezer.
The freezer operates nearly 24 hours per day, and on average it draws 200 W. 
Ice bricks and frozen drinks produced by the freezer are sold to neighboring villages and are a significant source of income for the community.
In addition to powering the freezer, the micro-grid serves approximately 20 households.
Each household is equipped with two 5 W LED light bulbs and a two plug outlet. 
The outlets are used primarily for cellphone charging.
A few households may also have larger electronics, such as a television or radio.
All households are individually metered, resulting in a financially motivated culture of energy conservation. 
Thus, residential electricity peak demand is significantly less than the maximum possible power draw. 
On average, residential electricity demand peaks during the early evening at 150 W.
The daily average of the demand data used in our analysis is illustrated as the dashed line in figure \ref{HourBar}.

When conducting our financial analysis, we estimated the annual cost of PV capacity to be 0.1762 \unit{USD/W}.
We also estimated the annual cost of battery capacity to be 0.0804 \unit{USD/ W  \! \cdot \! hr}. 
For a more detailed explanation of how the weather, demand, and economic parameters were generated, please refer to Appendix \ref{A2}.

\subsection{Case Study Procedure and Results}

% Introduce Sub-Section

Our design procedure for the micro-grid in Segou, Mali, required several iterations.
Each iteration generated a unique micro-grid configuration. 
The \emph{LEGP}, PV capacity, battery capacity, and cost per \unit{kW \! \cdot \! hr} of select iterations are printed in table \ref{ConfigSumm}.

% Design alternative summary table

\begin{table}

\centering
\begin{tabular}{|l|c|c|c|c|c|}
\hline
\emph{LEGP}  &  PV Capacity & Bat Capacity & Total Cost\\
      & \unit{W}     & \unit{W \! \cdot \!hr} & \unit{USD/kW \! \cdot \!hr} \\
\hline
Current Grid & & & \\
\emph{LEGP} = 0.054 & 1400 & 17280 & 0.881 \\
\hline
0.05 & 2000 & 7800 & 0.524 \\
\hline
0.03 & 2600 & 8400 & 0.591 \\
\hline
0.01 & 2500 & 13000 & 0.763 \\
\hline
\end{tabular}
\caption[Caption for LOF]{Performance, size, and costs characteristics of current and potential micro-grid's for Segou, Mali.\footnotemark}
\label{ConfigSumm}
\end{table}

\footnotetext{Note that the \emph{LEGP} and the cost per \unit{kW\! \cdot \! hr} were estimated by our model and do not constitute measured reliability or cost data.}
% 0.05 LEGP Micro-Grid

When sizing the micro-grid for the village outside of Segou, we started by specifying an annual \emph{LEGP} of 0.05.
Recognizing the severe financial limitations on micro-grid design, we wanted to start with a reliability that was marginally better than that of the current system.
After specifying an annual \emph{LEGP} of 0.05, we plotted the \emph{LEGP} for each operational month. 
Figure \ref{MonthBar} illustrates that, as expected from the weather patterns of Segou, the months of July through September had the lowest reliability.  
Because it had the lowest reliability, with an \emph{LEGP} of 0.136, we chose to analyze the month of August with an hourly resolution. 

When analyzing the month of August with an hourly resolution, we created the figures described in section 2.
After analyzing these two figures, we believe that the 0.05 \emph{LEGP} micro-grid does not offer sufficient performance.
The micro-grid fails to meet the needs of the 20 households.
Figure \ref{HourBar} illustrates that for the hours of significant residential demand, 7:00 PM to 1:00 AM, reliability steadily decreases. 
During this time, the \emph{LEGP} steadily increases from 0.052 to 0.205. 
During the peak demand window, which occurs between 9:00 PM until 10:00 PM, the \emph{LEGP} of the micro-grid was 0.138.
The \emph{LEG} map within figure \ref{LEGMaps} confirms the inconveniences to the micro-consumers.
In particular, we can see that the high \emph{LEGPs} were the result of regularly occurring energy shortages and not isolated outages.
Figure \ref{LEGMaps} indicates that there were five energy shortfall events which curtailed demand during the 7:00 PM to 1:00 AM period.
Four of these inhibit electricity consumption during the hour of peak demand, 9:00 PM to 10:00 PM.
We also found that the micro-grid performance was equally unsatisfactory for the owners of the freezer.
We estimate that any energy shortfalls lasting five hours or longer would significantly impact the production of ice and frozen drinks. 
As figure \ref{LEGMaps} illustrates, energy shortfalls lasting five or more hours occurred prior to, or during, eight separate work days.
 
% Now it is on to do a brief summary 
% of the LEGP = 0.03 microgrid and Full summary of why 
% the LEGP = 0.01 micro-grid is wonderful.

Recognizing that an \emph{LEGP} of 0.05 was insufficient to meet our consumers' demand, we iteratively increased system reliability and observed the temporal characteristics using the aforementioned methodology.
An intermediate design was of a micro-grid with an annual \emph{LEGP} of 0.03. 
We found that the \emph{LEGP} micro-grid was not sufficient and that its inadequacies closely paralleled those of the 0.05 \emph{LEGP} micro-grid.
As indicated by figure \ref{MonthBar}, August was again the month with the lowest reliability; thus, we analyzed this period with an hourly temporal resolution.
Our results indicate that this design would have a significant negative impact on residential electricity usage.
Recognizing that most residential demand occurs between 7:00 PM and 1:00 AM, figure \ref{LEGMaps} illustrates that there were four instances in which power outages would significantly curtail residential demand. 
With respect to the freezer operators, we found that there would be five outages which have a duration of five hours or longer. 
Subjectively, we determined that five days of lost revenue concentrated within a one month period would be unacceptable to the freezer operators. 

As a result of our iterative design process, we decided upon a micro-grid with an annual \emph{LEGP} of 0.01. After isolating the lowest reliability month, and analyzing it with an hourly resolution, we decided that the micro-grid was acceptable to residential consumers. 
%
% footnote
%
\footnote{Observing figure \ref{MonthBar}, September has a slightly higher \emph{LEGP} than August for the 0.01 annual \emph{LEGP} micro-grid. 
We present the hourly performance of August to facilitate reader comparison of design iterations. 
Although not presented, we also analyzed September with an hourly resolution, and the results confirmed our findings.}
%
%
%
Figure \ref{LEGMaps} indicates that there was only a single energy shortage between 7:00 PM and 1:00 AM, representing a single event in which residential consumption was significantly impacted.
We also found that the 0.01 \emph{LEGP} micro-grid significantly improved freezer up time, especially when compared to the aforementioned alternatives. There were only two energy shortfall occurrences which were five ours or longer. 
Given the reduced frequency and duration of energy shortfalls, it is possible that their effects can be further reduced by demand side energy management.

\section{Discussion}

\appendix	
\section{Appendix 1: \emph{LEGP} and Energy Balance Algorithm} \label{A1}

We chose this metric for two reasons.
First, our procedure allows use understand both the frequency and magnitude of energy shortfalls. 
Second, this metric is analogous to maximum annual capacity shortage which is a metric used by HOMER.


\section{Appendix 2: Explaination of Case Study Input Parameters} \label{A2}

\subsection{Weather Data}
\subsection{Demand Data}
\subsection{Cost Parameters}

\begin{thebibliography}{9}

\bibitem{cite1}

\end{thebibliography}
\end{document}