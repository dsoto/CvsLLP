\documentclass{article}
\usepackage{amsmath}
\usepackage{amssymb}
\usepackage{graphicx}
\usepackage[top=1 in, bottom=1 in, left=1 in, right=1 in]{geometry}
\newcommand{\unit}[1]{\ensuremath{\, \mathrm{#1}}}

\begin{document}

\begin{abstract}

\end{abstract}

%% introduction
    %% emphasizes need for storage on smaller grids
    %% emphasizes need to reduce storage costs
    %% introduce cost vs. reliability tradeoff
    %% outlines sections of the paper

%% methodology
    %% explain goal of creating cost vs. reliability curves
    %% introduce assumptions and inputs (tell what you are going to tell)
        %% year of demand
        %% year of supply
        %% basic energy balance
        %% lowest cost system at each reliability level
        %% table of assumptions?
    %% year of demand paragraph
    %% year of supply paragraph
    %% teaching figure - demand and supply
    %% energy balance allows reliability calculation
        %% use energy balance to LEG
        %% explain LEG -> LEGP
    %% calculate over pv-batt space
    %% teaching figure - iso-reliability curve
    %% main figure - cost vs. reliability curve
        %% salient features of curve
        %% day-night (shallow slope) vs weather (steep)
        
%% Discussion Section
	%% Do an overview of discussion (Tell what I am going to tell)
		%% Cost vs reliability curves do not tell whole story
		%% Temporal Behavior of micro-grid is important
			%% Hourly reliability of the average day
			%% Reliability of day changes with the season
		%% - The conclusions drawn about this micro-grid are unique to this
		%% - micro-grid. The decision making processes will facilitate the 	
		%% - sizing of micro-grids by others.
		
	%% Effects of load profile on  cost versus LEGP curve
	%% Temporal Behavior of average day for micro-grids.
		%% - Present cost versus LEGP curve for micro-grid without fridgerator
		%% - base load. 
		%% ** Present bar plot of LEGP per hour for the micro-grid in Mali
		%% ** with and without refrigerator base load. Include bars for LEGP
		%% ** of 0.01 and 0.05. 
		%% - Describe based on the information present thus far, what decisions  
		%% - I would make for sizing the micro-grid.
	%% Seaonal temporal behavior of LEGP for micro-grids. 
		%% - Present the bar plots of LEGP vs Month. Do the plots for the 
		%% - micro-grid with and without the fridgerator base load side 
		%% - by side. 
		%% ** Do one paragraph on each demand profile. Decide whether or not 
		%% ** the decisions that I already made still make sense. If not 
		%% ** what is the new decision that I would make. 
	

\section{Introduction}

As the size of electricity distribution networks is decreased,
so is the diversity of demand sources and 
the diversity of generation.
Thus, the smaller an electricity distribution grid, the more vulnerable 
it becomes to fluctuations in demand and electricity generation.
As a result, small standalone micro-grids must rely more heavily on
energy storage in order to buffer variability and meet demand.  
The inclusion of storage capacity greatly increases the capital cost of micro-grids.
Thus, it is vital to appropriately size electricity storage capacity.

For communities which are geographically isolated and lack economical development, standalone micro-grids often represent the only potential form of affordable and reliable electricity service. 
Given cost constraints, it may be desirable to make trade-offs between system costs and reliability for these communities. 
In making such trade-offs, storage capacity would be undersized in a manner which 
drastically %significantly
reduces capital costs, minimally impairs economic activity, and is acceptable to micro-grid consumers. 
%we don't have to be as narrow as "economic activity."

This paper is focused upon cost versus reliability trade-offs in
the context of standalone micro-grids with PV electricity generation
and battery storage. 
In the first section of the article we describe our methodology 
for analyzing micro-grids. 
We provide an overview of the computational tools that we have developed.
We also explain the metric that we use to quantify micro-grid reliability
and is advantages.
In the second section of the article, we characterize the relationship 
between cost and reliability for stand alone PV micro-grids. 
In particular, we analyze the effects of insolation availability on 
the interplay between cost in reliability. 
%(Im starting to feel like that
% we can produce a coherent 5 page paper without this section)
In the third section of the article, we critically evaluate the 
the effects of specified reliability on a micro-grid's temporal performance 
characteristics. 
Temporal behavior is evaluated on two different time scales. 
We observe micro-grid performance on a daily level with hourly resolution. 
Micro-grid performance is also evaluated on a yearly level 
with monthly resolution. 

\section{Methodology}
\subsection{Overview}

Our analysis of standalone micro-grids is hinged upon the creation of cost versus reliability curves.
These curves are useful because they illuminate the marginal cost of system reliability. 
In doing so, they indicate when adding additional micro-grid storage capacity will have diminishing returns.
In order to create these cost versus reliability curves, we require one year of micro-grid demand data and insolation data with an hourly resolution.
This data is then utilized by an energy balance algorithm.
From the energy balance algorithm we learn the battery charge state at all hours. 
We also become aware of the magnitude and instances of all energy shortfalls. 
The energy balance algorithm is then utilized by an optimization algorithm which computes all PV and battery combinations which achieve a specified reliability. 
The most affordable combination and the associated cost is stored by the algorithm. 
The process is repeated for several reliabilities, and a cost versus reliability curve is populated. 

\subsection{Input Parameters}
% Origin of demand data
%
The electricity demand data utilized by the energy balance algorithm is from a micro-grid that we installed near Segou,Mali. 
The battery bank's charge state is measured in {\unit{W\!\cdot \! hr}} for each hour of the year.
The micro-grid supplies electricity to twenty households which use electricity primarily for nighttime lighting.
Also on the micro-grid is a 250 liter freezer which provides the micro-grid with a base demand of approximately 250 W. 
Peak demand on the micro-grid is approximately 450 W. 
A single week of demand data was created by combining seven unique days without energy shortfalls. 
The annual consumer demand data was created by propagating the week of demand over an entire year.

% Origin of resource data
% 
Unlike the demand data, our solar resource data does not originate from on site measurements.
For isolated locations in the developing world, ground based insolation data is not usually available. 
When available this data often lacks sufficient temporal resolution. 
Thus, all insolation measurements were extrapolated from the HelioClim 3 database.
HelioClim 3 data is derived from Meteosat Second Generation satellite images.
The utilized HelioClim 3 data is of normal to sun insolation on the ground level.
Gaps within the Helioclim database were filled by using the corresponding hours in the preceding day or days.
None of insolation data sets utilized contained more than eight days of missing data.
In order to convert the HelioClim 3 normal to sum insolation, $I_B$, into the insolation on the collector,$I_C$, the methodology by Masters \cite{} was used. 
$I_C$ included three components: direct beam ($I_{BC}$), diffuse ($I_{DC}$), and ground reflected ($I_{RC}$).
For all calculations, the collector tilt, $\Sigma$, was equal to the latitude of the weather profile location, and the ground reflectance $\rho$ was set to 0.20.
A summary of the input parameters for solar resource calculations in included as table \ref{}.
% Make a table detailing assumptions for weather data

%Optimization Parameters
Throughout all of the analysis we used a fixed capital cost of 1.50 \unit{USD/W} for PV capacity and a cost of 0.20 \unit{USD/W \! \cdot \! hr} for battery capacity.
The maximum allowable battery depth of discharge, DOD, is equal to 50 percent, effectively doubling the cost of energy storage to 0.40 \unit{USD/W \! \cdot \! hr}. 
The payback period for the PV modules was 20 years, and the payback period for the batteries is three years. 
We applied a ten percent annual compound interest to the capital costs. 
A summary of the cost analysis input parameters is provided within table \ref{}.

\subsection{Optimization Parameters and Strategy}

As previously stated, the efficacy of a micro-grid is assessed through the use of an energy balance algorithm with an hourly time scale. 
In order to conduct the one year simulation,the maximum and minimum allowable battery charge levels $E_{Bmin}$ and $E_{Bmax}$ must be specified in \unit{W\! \cdot \! hr}.
We then compute the charge level of the battery bank for all hours using equation \ref{eq:ESUM}. 

\begin{equation} \label{eq:ESUM}
E_B(t+1) =
\begin{cases}
E_{Bmax} & \text{if } E_B(t+1) \geq E_{Bmax},\\
E_B (t) + E_{PV} (t) - E_{dem} (t) & \text{if } E_{Bmin}\geq E_B(t+1)\leq E_{Bmax},\\
E_{Bmin} & \text{if } E_B(t) \leq E_{Bmin}.
\end{cases}
\end{equation}

Within equation \ref{eq:ESUM}, $E_B(t)$ is the energy contained within the battery bank, $E_{PV}(t)$ is the energy generated by the PV array, and $E_{dem}(t)$ is the energy demanded by the micro-grid consumers.
Represented as figure \ref{energyBalance} is an example of week's results for the energy balance algorithm. 
\begin{figure}[ht] 
  \centering
    \includegraphics[trim = 12mm 50mm 12mm 47mm, clip,width=.75\textwidth]{energyBalance.pdf}
  \caption{Energy balance algorithm calculations from January 2, 2005 until Jan 8, 2005.
Simulation uses micro-grid demand with refrigerator base load and insolation profile from Segou, Mali. 
Micro-grid contains 2500 \unit{W} of nominal PV capacity, and 13000 \unit{W\! \cdot \! hr} of nominal battery capacity. 
Battery low charge disconnect is 6500 \unit{W\! \cdot \! hr}. 
Y axis scale for electricity supply and demand is on the left side of the figure.
Y axis scale for battery storage is on the right of figure.}
\label{energyBalance}
\end{figure}
When the battery capacity has been depleted, we quantify the demand that cannot be met using a variable called lack of energy to generate, $LEG$, which was proposed by Wissem et al. \cite{ }. 
We define $LEG$ using equation \ref{eq:LEG}. 

\begin{equation} \label{eq:LEG}
LEG(t) = E_{dem} (t) - (E_{pv} (t)+E_B(t-1)-E_{Bmin})
\end{equation}

We are then able to assess the micro-grid's performance using a metric called lack of energy to generate probability, $LEGP$, which was also proposed by Wissem et al. \cite{ }.
\emph{LEGP} is equal to the electricity demand that a micro-grid was unable to meet divided by the total demand over a specified time frame.
It is calculated using equation \ref{eq:LEGP}.

\begin{equation} \label{eq:LEGP}
LEGP = \frac{\sum_{t=1}^T LEG(t)}{\sum_{t=1}^T E_{dem} (t)}
\end{equation}

Although time based metrics such as loss of load probability \emph{LOLP} may be used to evaluate system performance, these tend to inflate the perceived performance of micro-grids in the developing world.
Many of the micro-grids we have observed are only used for a small fraction of the day.
We are more concerned about a micro-grids ability to supply demanded energy, and thus chose \emph{LEGP} as our primary performance metric.

We decided to quantify system cost based on the "break-even" consumer price of electricity in cost per kilowatt-hour\unit{(USD/kW\! \cdot \! hr)}.
As opposed to installation capital cost, or annual payment, this metric allows us to directly compare micro-grids of different sizes and demand patterns.
Moreover, the metric is easily understood by the development community, and it allows for electricity rates to be easily compared to those of traditional grid systems.
 
% Describe the isoreliability cost versus LEGP Curves
We decided that our cost versus reliability curve would be expressed as cost per \unit{kW \! \cdot \! hr} versus \emph{LEGP}.
In order to create this curve, we must know the lowest cost PV and battery combination for each desired \emph{LEGP}.
To find the lowest cost options, we create PV versus battery curves.
These curves contain all PV and Battery combinations which maintain the desired \emph{LEGP}. 
We then compute the cost of each point on the iso-reliability curve.
An example of an iso-reliability curve is illustrated as figure \ref{energyBalance}.
%
%insert a figure containing an iso-reliability plot and the associated cost curve
%
\begin{figure}[ht] 
  \centering
    \includegraphics[trim = 25mm 86mm 25mm 90mm, clip,width=.8\textwidth]{isoReliabilityCurve.pdf}
  \caption{Iso-reliability curve of a micro-grid system with an \emph{LEGP} of 0.05. 
Simulation uses micro-grid demand with refrigerator base load and insolation profile from Segou, Mali.
Superimposed on the plot is the cost per kW-hr of each micro-grid combination.
The minimum cost point for both curves is marked with an x.}
\label{energyBalance}
\end{figure}
%
%
Once we have computed the lowest cost PV and battery combination for each of the desired \emph{LEGPs}, we are able to assemble a cost in \unit{USD/kWh\! \cdot \!hr} curve. 
An example of cost versus reliability curves is illustrated as figure \ref{MaliFandLCostPerkWhVsLEGP}

\begin{figure}[ht] 
  \centering
    \includegraphics[trim = 25mm 85mm 25mm 90mm, clip,width=.75\textwidth]{MaliFandLCostPerkWhVsLEGP.pdf}
  \caption{Cost in \unit{USD/kw\! \cdot \! hr} versus \emph{LEGP} of 
micro-grid with refrigerator and lighting only base loads.
The simulation uses the insolation profile from Segou, Mali.
\emph{LEGPs} range from 0.01 to 0.20}
\label{MaliFandLCostPerkWhVsLEGP}
\end{figure}


% Explain the main features of the cost versus reliabilty curve

The cost versus \emph{LEGP} curves illustrated as figure \ref{MaliFandLCostPerkWhVsLEGP} are of a micro-grid with and without a refrigerator base load. The weather profile used by the model is of Segou, Mali.
When analyzing figure \ref{MaliFandLCostPerkWhVsLEGP} for the micro-grid with a refrigerator base load, we observe a knee in the curve where \emph{LEGP} is approximately equal to 0.05.
For \emph{LEGPs} greater than 0.05, adding micro-grid capacity has a high return on investment; for a small increase in the cost of electricity, system reliability is greatly increased. 
This is because added capacity is reducing diurnal energy shortfalls which occur prior to sunrise. 
For lower \emph{LEGPs}, the marginal cost of added reliability increases. 
This is because added capacity is being used to remove energy shortfalls which occur as a result of seasonal rain and cloud coverage. 

%% Describe why the shape is different for the micro-grid without the base-load. 
	%% -  I am assuming this is because there isn't much electricity usage in
	%% - in the early morning.
	%% Do some testing with pretend data to verify or reject hypothesis

The curves within figure \ref{MaliFandLCostPerkWhVsLEGP} have a fundamentally different shape than the cost versus reliability curve created by Wissem et al. \cite{}. 
This is because, unlike Wissem et al. \cite{•}, we only apply a positive value to electricity which can be sold to the micro-grid.
Energy which cannot be sold directly to the micro-grid, and cannot be stored, is considered lost and does not generate revenue. 

\section{Decision Making Based on Temporal Characteristics}

Cost versus \emph{LEGP} curves, like that represented in figure \ref{MaliFandLCostPerkWhVsLEGP}, provide a means to understand the marginal cost of additional micro-grid capacity. 
By doing so they illustrate the break even cost of electricity, which must be charged to consumers, in order to provide a particular \emph{LEGP} of service. 
However, these curves do not indicate if the design \emph{LEGP} satisfies the consumers' electricity demand needs. 
In order to observe the efficacy of a potential micro-grid, we analyze the temporal performance of the grids.
This temporal analysis is conducted in two primary forms. 
In the first, we observe the mean hourly \emph{LEGP} for the entire year. 
In the second, we observe the seasonal variations in \emph{LEGP}.
We recognize that our work will not directly serve as a template or decision-tree for others.
However, we believe that presenting our decision processes and data visualization methods will help others appropriately size micro-grids.

\subsection{Hourly Variations in Micro-Grid Performance}

%% These figures Illustrate
	%% Hourly reliability and yearly reliability are not the same
	%% - The hourly reliability is strongly effected by the weather and demand
	%% - profiles.
	%% ** The imbalance between daytime and nighttime demand exacerbates the
	%% ** difference between daytime and nightime reliability.
	
%% These figures imply 
	%%  0.05 LEGP might be acceptible for the both micro-grids. 
	%% ** For the lighting only system there is almost no demand when there 
	%% ** is a large energy shortfall.
	%% - For the micro-grid with refrigerator energy shortfalls occur when it 
	%% - it is cool. 0.05 might be acceptible, as long as, the people keep the
	%% - door closed.
	%% ** In order to fully understand the micro-grid performance, we need to 
	%% ** analyze the data further, and see how performance varies by season. 

\begin{figure}[ht]
\centering
\includegraphics[trim = 0mm 35mm 0mm 35mm, clip,width=1\textwidth]{timeOfSFByHourFrLig0105}
\caption{Probability of energy shortfall for each hour. 
Superimposed on the plots is the energy demanded by the micro-grids. 
On the left is the results of \emph{LEGP} optimization for micro-grids with a refrigerator base load. 
For an annual \emph{LEGP} of 0.01, its battery and PV capacities are 13000 \unit{W \! \cdot \!hr} and 2500 W.
For an annual \emph{LEGP} of 0.05, its battery and PV capacities are 7800 \unit{W \! \cdot \!hr} and 2000 W.
On the right is the results of the same analysis without a refrigerator base load. 
For an annual \emph{LEGP} of 0.01, its battery and PV capacities are 3600 \unit{W \! \cdot \!hr} and 500 W.
For an annual \emph{LEGP} of 0.05, its battery and PV capacities are 2600 \unit{W \! \cdot \!hr} and 400 W.}
\label{hourlyLEGP}
\end{figure} 

\subsection{Seasonal Variations of Micro-Grid Performance}



\end{document}
