\documentclass{article}
\usepackage{amsmath}
\usepackage{amssymb}
\usepackage{graphicx}
\usepackage[top=1 in, bottom=1 in, left=1 in, right=1 in]{geometry}
\newcommand{\unit}[1]{\ensuremath{\, \mathrm{#1}}}

\begin{document}

\begin{abstract}

\end{abstract}

\section{Introduction}

As the size of electricity distribution networks is decreased,
so is the diversity of demand sources and 
the diversity of generation.
Thus, the smaller an electricity distribution grid, the more vulnerable 
it becomes to fluctuations in demand and electricity generation.
As a result, small standalone micro-grids must rely more heavily on
energy storage in order to buffer variability and meet demand.  
The inclusion of storage capacity greatly increases the capital cost of micro-grids.
Thus, it is vital to appropriately size electricity storage capacity.

For communities which are geographically isolated and lack economical development, standalone micro-grids often represent the only potential form of affordable and reliable electricity service. 
Given cost constraints, it may be desirable to make trade-offs between system costs and reliability for these communities. 
In making such trade-offs, storage capacity would be undersized in a manner which 
drastically %significantly
reduces capital costs, minimally impairs economic activity, and is acceptable to micro-grid consumers. 
%we don't have to be as narrow as "economic activity."

This paper is focused upon cost versus reliability trade-offs in
the context of standalone micro-grids with PV electricity generation
and battery storage. 
In the first section of the article we describe our methodology 
for analyzing micro-grids. 
We provide an overview of the computational tools that we have developed.
We also explain the metric that we use to quantify micro-grid reliability
and is advantages.
In the second section of the article, we characterize the relationship 
between cost and reliability for stand alone PV micro-grids. 
In particular, we analyze the effects of insolation availability on 
the interplay between cost in reliability. 
%(Im starting to feel like that
% we can produce a coherent 5 page paper without this section)
In the third section of the article, we critically evaluate the 
the effects of specified reliability on a micro-grid's temporal performance 
characteristics. 
Temporal behavior is evaluated on two different time scales. 
We observe micro-grid performance on a daily level with hourly resolution. 
Micro-grid performance is also evaluated on a yearly level 
with monthly resolution. 

\section{Methodology}
\subsection{Input Parameters}
Our analysis of standalone micro-grid performance draws upon the energy balance algorithm used by Wissem et al \ref{ }. 
Two data sets are required in order to use the energy balance algorithm, one of energy consumption, the other describing solar resource availability.
Energy usage and insolation data was used at hourly temporal resolution. 
% Origin of demand data
%
The electricity demand data is from a micro-grid that we installed near Segou,Mali and is measured in {\unit{W\!\cdot \! hr}} throughout the year. 
The micro-grid supplies electricity to twenty households which use electricity primarily for nighttime lighting.
Also on the micro-grid is a 250 liter freezer which provides the micro-grid with a base demand of approximately 250 W. 
Peak demand on the micro-grid is approximately 450 W. 
A single week of demand data was created by combining seven unique days without energy shortfalls. 
The annual consumer demand data was created by propagating the week of demand over an entire year.
% Origin of resource data
% 
For isolated locations in the developing world, ground based insolation data is not usually available. 
When available this data often lacks sufficient temporal resolution. 
Thus, all insolation measurements were extrapolated from the HelioClim 3 database.
HelioClim 3 data is derived from Meteosat Second Generation satellite images.
The utilized HelioClim 3 data is of normal to sun insolation on the ground level.
When gaps within the Helioclim database were filled by using the corresponding hours in the preceding day or days.
%How many gaps in the data were there?

%Optimization Parameters
Throughout all of the analysis we used a fixed capital cost of 1.50 \unit{USD/W} for PV capacity and a cost of 0.20 \unit{USD/W \! \cdot \! hr} for battery capacity.
The maximum allowable battery depth of discharge, DOD, is equal to 50 percent, effectively doubling the cost of energy storage to 0.40 \unit{USD/W \! \cdot \! hr}. 
The payback period for the PV modules was 20 years, and the payback period for the batteries is three years. 
We applied a ten percent annual compound interest to the capital costs. 
A summary of the cost analysis input parameters is provided within table \ref{}.

\subsection{Optimization Parameters and Strategy}


\end{document}
