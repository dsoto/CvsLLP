\documentclass{article}
\usepackage{amsmath}
\usepackage{amssymb}
\usepackage{graphicx}
\newcommand{\unit}[1]{\ensuremath{\, \mathrm{#1}}}

\begin{document}

\begin{abstract}

\end{abstract}

\section{Introduction}

As the size of electricity distribution networks is decreased,
so is the diversity of demand sources and 
the diversity of generation.
Thus, the smaller an electricity distribution grid, the more vulnerable 
it becomes to fluctuations in demand and electricity generation.
As a result, small standalone micro-grids must rely more heavily on
energy storage in order to buffer variability and meet demand.  
The inclusion of storage capacity greatly increases the capital cost of micro-grids.
Thus, it is vital to appropriately size electricity storage capacity.

For communities which are geographically isolated and lack economical development, standalone micro-grids often represent the only potential form of affordable and reliable electricity service. 
Given cost constraints, it may be desirable to make trade-offs between system costs and reliability for these communities. 
In making such trade-offs, storage capacity would be undersized in a manner which 
drastically reduces capital costs, minimally impairs economic activity, and is acceptable to micro-grid consumers. 

This paper is focused upon cost versus reliability trade-offs in
the context of standalone micro-grids with PV electricity generation
and battery storage. 
In the first section of the article we describe our methodology 
for analyzing micro-grids. 
We provide an overview of the computational tools that we have developed.
We also explain the metric that we use to quantify micro-grid reliability
and is advantages.
In the second section of the article, we characterize the relationship 
between cost and reliability for stand alone PV micro-grids. 
In particular, we analyze the effects of insolation availability on 
the interplay between cost in reliability. 
%(Im starting to feel like that
% we can produce a coherent 5 page paper without this section)
In the third section of the article, we critically evaluate the 
the effects of specified reliability on a micro-grid's temporal performance 
characteristics. 
Temporal behavior is evaluated on two different time scales. 
We observe micro-grid performance on a daily level with hourly resolution. 
Micro-grid performance is also evaluated on a yearly level 
with monthly resolution. 

\section{Methodology}

\end{document}
