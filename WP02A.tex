\documentclass{article}
\usepackage{amsmath}
\usepackage{amssymb}
\usepackage{graphicx}
\usepackage[top=1 in, bottom=1 in, left=1 in, right=1 in]{geometry}
\newcommand{\unit}[1]{\ensuremath{\, \mathrm{#1}}}

\begin{document}

\begin{abstract}

\end{abstract}

\section{Introduction}

As the size of electricity distribution networks is decreased,
so is the diversity of demand sources and 
the diversity of generation.
Thus, the smaller an electricity distribution grid, the more vulnerable 
it becomes to fluctuations in demand and electricity generation.
As a result, small standalone micro-grids must rely more heavily on
energy storage in order to buffer variability and meet demand.  
The inclusion of storage capacity greatly increases the capital cost of micro-grids.
Thus, it is vital to appropriately size electricity storage capacity.

For communities which are geographically isolated and lack economical development, standalone micro-grids often represent the only potential form of affordable and reliable electricity service. 
Given cost constraints, it may be desirable to make trade-offs between system costs and reliability for these communities. 
In making such trade-offs, storage capacity would be undersized in a manner which 
drastically %significantly
reduces capital costs, minimally impairs economic activity, and is acceptable to micro-grid consumers. 
%we don't have to be as narrow as "economic activity."

This paper is focused upon cost versus reliability trade-offs in
the context of standalone micro-grids with PV electricity generation
and battery storage. 
In the first section of the article we describe our methodology 
for analyzing micro-grids. 
We provide an overview of the computational tools that we have developed.
We also explain the metric that we use to quantify micro-grid reliability
and is advantages.
In the second section of the article, we characterize the relationship 
between cost and reliability for stand alone PV micro-grids. 
In particular, we analyze the effects of insolation availability on 
the interplay between cost in reliability. 
%(Im starting to feel like that
% we can produce a coherent 5 page paper without this section)
In the third section of the article, we critically evaluate the 
the effects of specified reliability on a micro-grid's temporal performance 
characteristics. 
Temporal behavior is evaluated on two different time scales. 
We observe micro-grid performance on a daily level with hourly resolution. 
Micro-grid performance is also evaluated on a yearly level 
with monthly resolution. 

\section{Methodology}
\subsection{Input Parameters}
Our analysis of standalone micro-grid performance draws upon the energy balance algorithm used by Wissem et al \ref{ }. 
Two data sets are required in order to use the energy balance algorithm, one of energy consumption, the other describing solar resource availability.
Energy usage and insolation data was used at hourly temporal resolution. 
% Origin of demand data
%
The electricity demand data is from a micro-grid that we installed near Segou,Mali and is measured in {\unit{W\!\cdot \! hr}} throughout the year. 
The micro-grid supplies electricity to twenty households which use electricity primarily for nighttime lighting.
Also on the micro-grid is a 250 liter freezer which provides the micro-grid with a base demand of approximately 250 W. 
Peak demand on the micro-grid is approximately 450 W. 
A single week of demand data was created by combining seven unique days without energy shortfalls. 
The annual consumer demand data was created by propagating the week of demand over an entire year.

% Origin of resource data
% 
For isolated locations in the developing world, ground based insolation data is not usually available. 
When available this data often lacks sufficient temporal resolution. 
Thus, all insolation measurements were extrapolated from the HelioClim 3 database.
HelioClim 3 data is derived from Meteosat Second Generation satellite images.
The utilized HelioClim 3 data is of normal to sun insolation on the ground level.
Gaps within the Helioclim database were filled by using the corresponding hours in the preceding day or days.
None of insolation data sets utilized contained more than eight days of missing data.
In order to convert the HelioClim 3 normal to sum insolation, $I_B$, into the insolation on the collector,$I_C$, the methodology by Masters \cite{} was used. 
$I_C$ included three components: direct beam ($I_{BC}$), diffuse ($I_{DC}$), and ground reflected ($I_{RC}$).
For all calculations, the collector tilt, $\Sigma$, was equal to the latitude of the weather profile location, and the ground reflectance $\rho$ was set to 0.20.
A summary of the input parameters for solar resource calculations in included as table \ref{}.
% Make a table detailing assumptions for weather data

%Optimization Parameters
Throughout all of the analysis we used a fixed capital cost of 1.50 \unit{USD/W} for PV capacity and a cost of 0.20 \unit{USD/W \! \cdot \! hr} for battery capacity.
The maximum allowable battery depth of discharge, DOD, is equal to 50 percent, effectively doubling the cost of energy storage to 0.40 \unit{USD/W \! \cdot \! hr}. 
The payback period for the PV modules was 20 years, and the payback period for the batteries is three years. 
We applied a ten percent annual compound interest to the capital costs. 
A summary of the cost analysis input parameters is provided within table \ref{}.

\subsection{Optimization Parameters and Strategy}

As previously stated, the efficacy of a micro-grid is assessed through the use of an energy balance algorithm with an hourly time scale. 
In order to conduct the one year simulation,the maximum and minimum allowable battery charge levels $E_{Bmin}$ and $E_{Bmax}$ must be specified in \unit{W\! \cdot \! hr}.
We then compute the charge level of the battery bank for all hours using equation \ref{eq:ESUM}. 

\begin{equation} \label{eq:ESUM}
E_B(t+1) =
\begin{cases}
E_{Bmax} & \text{if } E_B(t+1) \geq E_{Bmax},\\
E_B (t) + E_{PV} (t) - E_{dem} (t) & \text{if } E_{Bmin}\geq E_B(t+1)\leq E_{Bmax},\\
E_{Bmin} & \text{if } E_B(t) \leq E_{Bmin}.
\end{cases}
\end{equation}

Within equation \ref{eq:ESUM}, $E_B(t)$ is the energy contained within the battery bank, $E_{PV}(t)$ is the energy generated by the PV array, and $E_{dem}(t)$ is the energy demanded by the micro-grid consumers.
When the battery capacity has been depleted, we quantify the demand that cannot be met using a variable called lack of energy to generate, $LEG$, which was proposed by Wissem et al. \cite{ }. 
We define $LEG$ using equation \ref{eq:LEG}. 

\begin{equation} \label{eq:LEG}
LEG(t) = E_{dem} (t) - (E_{pv} (t)+E_B(t-1)-E_{Bmin})
\end{equation}

We are then able to assess the micro-grid's performance using a metric called lack of energy to generate probability, $LEGP$, which was also proposed by Wissem et al. \cite{ }.
\emph{LEGP} is equal to the electricity demand that a micro-grid was unable to meet divided by the total demand over a specified time frame.
It is calculated using equation \ref{eq:LEGP}.

\begin{equation} \label{eq:LEGP}
LEGP = \frac{\sum_{t=1}^T LEG(t)}{\sum_{t=1}^T E_{dem} (t)}
\end{equation}

Although time based metrics such as loss of load probability \emph{LOLP} may be used to evaluate system performance, these tend to inflate the perceived performance of micro-grids in the developing world.
Many of the micro-grids we have observed are only used for a small fraction of the day.
We are more concerned about a micro-grids ability to supply demanded energy, and thus chose \emph{LEGP} as our primary performance metric.

We decided to quantify system cost based on the "break-even" consumer price of electricity in cost per kilowatt-hour\unit{(USD/kW\! \cdot \! hr)}.
As opposed to installation capital cost, or annual payment, this metric allows us to directly compare micro-grids of different sizes and demand patterns.
Moreover, the metric is easily understood by the development community, and it allows for electricity rates to be easily compared to those of traditional grid systems.
% This may be a good place to end this section (discuss with Daniel)
 
% Describe the isoreliability cost versus LEGP Curves
In order to assess how increasing system reliability drives system costs, we create cost \unit{USD/kW\! \cdot \! hr} versus \emph{LEGP} curves.
Each point on a \unit{USD/kW\! \cdot \! hr} versus \emph{LEGP} curve represents the lowest cost battery storage and PV generation combination for a specified reliability. 
The minimum cost combinations were created using our optimization strategy. 
A majority of the analysis we conducted may be replicated using \emph{HOMER} or another proprietary energy balance micro-grid simulator.
However, we chose to create our own energy balance and optimization models.
This gave us increased control over the modeling and optimization process.
In addition, it allowed us to not treat the software like a "black box."
The model we created contains two basic levels.
The inner level conducts an energy balance analysis on an hourly timescale.
This analysis simulates the production, storage, distribution, and consumption of energy by a micro-grid for each hour over the course of a year.
The energy balance level is called upon by an outer-level script which computes all of the PV generation and battery storage combinations which satisfy a particular LEGP.
This script then determines the generation and storage combination with the minimum initial capital investment cost.
The optimization is then executed for several LEGPs, resulting in a cost vs. reliability plot. 


\end{document}
