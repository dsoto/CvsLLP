\documentclass{article}
\usepackage{amsmath}
\usepackage{amssymb}
\usepackage{graphicx}
\newcommand{\unit}[1]{\ensuremath{\, \mathrm{#1}}}

\begin{document}

\begin{abstract}

\end{abstract}

\section{Introduction}

Within this article we describe thought processes and decision making strategies which may be employed in the sizing of isolated micro-grids for the developing world. 
Recognizing the sever costs constraints which result from consumers' financial limitations, we challenge the traditional western goal of 100 percent up-time. 
Instead, when sizing isolated micro-grids, policy makers must have a qualitative and quantitative understanding of costs versus reliability trade-offs.
This includes understanding the temporal reliability characteristics of a proposed micro-grid.
Knowing \emph{when} energy shortfalls occur will illuminate whether a proposed micro-grid matches consumer demand, and thus represents a satisfactory energy solution.  

This article is divided into four primary sections.
The first section details our procedure for generating cost versus reliability curves for isolated micro-grids. We quantify system reliability using an energy based metric called lack of energy to generate probability (\emph{LEGP}) introduced by Wissem et al.\cite{}. 
The cost versus \emph{LEGP} curves are based off of an energy balance algorithm, and they utilize historical hourly insolation and demand profiles.
The second section uses our energy balance algorithm to assess the effects of weather and demand profiles on cost versus system \emph{LEGP} curves.  
The third section focuses analyzes a particular insolation and demand profile. 




\end{document}