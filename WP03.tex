\documentclass{article}
\usepackage{amsmath}
\usepackage{amssymb}
\usepackage{graphicx}
\newcommand{\unit}[1]{\ensuremath{\, \mathrm{#1}}}

\begin{document}

\begin{abstract}

\end{abstract}

\section{Introduction}

Within this article we describe thought processes and decision making strategies which may be employed in the sizing of isolated micro-grids for the developing world.
Recognizing the sever costs constraints which result from consumers' financial limitations, we challenge the traditional western goal of 100 percent up-time.
% can we find evidence of this 100% approach being taken in the dev
% world? 
Instead, when sizing isolated micro-grids, policy makers must have a qualitative and quantitative understanding of costs versus reliability trade-offs.
This includes understanding the temporal reliability characteristics of a proposed micro-grid.
Knowing \emph{when} energy shortfalls occur will illuminate whether a proposed micro-grid matches consumer demand, and thus represents a satisfactory energy solution.

This article is divided into four primary sections.
The first section details our procedure for generating cost versus reliability curves for isolated micro-grids.
We quantify system reliability using an energy based metric called lack of energy to generate probability (\emph{LEGP}) introduced by Wissem et al.\cite{}.
The cost versus \emph{LEGP} curves are based off of an energy balance algorithm, and they utilize historical hourly insolation and demand profiles.
The second section uses our energy balance algorithm to assess the effects of weather and demand profiles on cost versus system \emph{LEGP} curves.
In the third section we investigate a particular insolation and demand profile, and we adjust the micro-grid's \emph{LEGP}.
In doing so, we assess the temporal behavior of decreased system reliability.
In the fourth, and final, section of the article, we further elucidate our findings and the lessons learned from our research.

\section{Generation of Cost Versus Reliability Curves}
% goal of section is to teach the reader about our general approach to
% reliability curves. (ADDED BELOW)
Within this section we provide a high-level description of our approach to generating cost versus reliability curves.
The primary metric we use to evaluate a micro-grid's performance is called lack of energy to generate probability (LEGP) which was introduced by Wissem et al.\cite{}.
LEGP is equal to the electricity demand that a micro-grid was unable to meet divided by the total demand over a specified time frame.
For our model the time frame was 8760 hours, or one year.
% Should we demonstrate this point with a small example?
Although time based metrics such as loss of load probability \emph{LOLP} may be used to evaluate system performance, these tend to inflate the perceived performance of micro-grids in the developing world.
Many of the micro-grids we have observed are only used for a small fraction of the day.
We are more concerned about a micro-grids ability to supply demanded energy, and thus chose \emph{LEGP} as our primary performance metric.

We decided to quantify system cost based on the "break-even" consumer price of electricity in cost per kilowatt-hour\unit{(USD/kWh)}.
As opposed to installation capital cost, or annual payment, this metric allows us to directly compare micro-grids of different sizes and demand patterns.
%% Talk about how we estimated costs (Cost estimation added below)
Throughout all of the analysis we used a fixed capital cost of 1.50 \unit{USD/W} for PV capacity and a cost of 0.20 \unit{USD/Whr} for battery capacity.
The payback period for the PV modules was 20 years, and the payback period for the batteries is three years. 
We applied a ten percent annual compound interest to the capital costs.

A majority of the analysis we conducted may be replicated using \emph{HOMER} or another proprietary energy balance micro-grid simulator.
However, we chose to create our own energy balance and optimization models.
This gave us increased control over the modeling and optimization process.
In addition, it allowed us to not treat the software like a "black box."
The model we created contains two basic levels.
The inner level conducts an energy balance analysis on an hourly timescale.
This analysis simulates the production, storage, distribution, and consumption of energy by a micro-grid for each hour over the course of a year.
The energy balance level is called upon by an outer-level script which computes all of the PV generation and battery storage combinations which satisfy a particular LEGP.
This script then determines the generation and storage combination with the minimum initial capital investment cost.
The optimization is then executed for several LEGPs, resulting in a cost vs. reliability plot.

\section{Impact of Weather and Demand Profiles on Cost vs. \emph{LEGP} Curve}
% goal of section is to demonstrate the impact of climate type on the
% marginal cost of reliability.

The goal of this section is to demonstrate the impact of climate type on the marginal cost of micro-grid reliability.
The results presented were obtained by using the demand profile of a micro-grid near Segou, Mali. 
To assess the effects of weather on the \emph{LEGP} vs. cost behavior, we paired this demand profile with three distinct weather data sets. 
Each data set is characteristic of a different climate type. 
The first is the semi-arid climate of Segou, Mali. This weather pattern has two distinct seasons, a rainy season and a dry season.
The dry season lasts from June until September with August being the month with the least solar availability. %%What about February ?
The second climate is the rain forest climate of Kisangani, Democratic Republic of the Congo. %% Go into more detail
The third climate is the desert climate of Luxor, Egypt. %% Go into more detail

There is a scarcity of hourly ground based insolation data for locations such as Segou, Mali, which are best served by isolated micro-grids. 
Thus, all insolation measurements were extrapolated from the HelioClim 3 database. The utilized HelioClim 3 data is of normal to sun insolation on the ground level.
HelioClim 3 data is derived from Meteosat Second Generation satellite images. 

The micro-grid being analyzed currently relies on 1.4 \unit{kW} of nominal PV capacity, and 17.280 \unit{kW\cdot hr} of lead acid battery storage.
The households use electricity primarily for nighttime lighting.
Also on the micro-grid is a 250 liter freezer which provides the micro-grid with a base demand of approximately 250 W. 
Peak demand on the micro-grid is approximately 450 W. 
The consumer demand data was created by propagating a single week of demand over an entire year.

\section{Impact of \emph{LEGP} on Micro-Grid Temporal Performance Characteristics}
% goal of section is to demonstrate that there is more to microgrid
% performance than a single metric.

% is there a difference between the per hour distribution for a
% continuous vs. a night only load?

Although \emph{LEGP} is an important metric, it by no means fully characterizes system performance. 
In particular, \emph{LEGP} does not describe when micro-grid energy shortfalls occur.
As we will illustrate, daily and seasonal variations in energy shortfall may greatly effect a micro-grid's viability for different demand profiles. 
We begin by analyzing the \emph{LEGP} curve generated using the Segou, Mali, weather data and the demand profile from the previous section. 
This plot is illustrated as figure \ref{CostVLEGPMali}.

\begin{figure}[ht]  
  \centering
    \includegraphics[trim = 10mm 70mm 10mm 70mm, clip,width=.85\textwidth]{MaliFridgeCostPerkWhVsLEGP}
  \caption{Cost vs. LEGP of Micro-Grid in Segou, Mali}
  \label{CostVLEGPMali}
\end{figure}

From figure \ref{CostVLEGPMali} we decided to further analyze the micro-grid temporal behavior with \emph{LEGPs} of 0.05 and 0.01.
 We chose to analyze $LEGP = 0.01$ because it approaches the upper threshold of consumer willingness to pay.
We chose to analyze the system characteristics with $LEGP = 0.05$ because it corresponds to intersection of the linear and exponential cost regimes.
According to figure \ref{CostVLEGPMali}, for \emph{LEGPs} below 0.05 the marginal cost of electricity generation begins to increase exponentially.  

%% Describe the PV and battery size combination for the LEGP 0.01 and the LEGP 0.05 systems. 
To observe the diurnal characteristics of the micro-grid's performance, we created figure \ref{hourlyLEGP}. Given our specified \emph{LEGPs} of 0.01 and 0.05, we are able to see the hourly \emph{LEGPs} for each potential system configuration.

\begin{figure}[ht]
\centering
\includegraphics[trim = 50mm 95mm 50mm 95mm, clip,width=1\textwidth]{timeOfShortfallByHourCount0105}
\caption{Probability of energy shortfall for each hour. Model uses weather from Segou, Mali, and micro-grid demand profile with refrigerator base load.}
\label{hourlyLEGP}
\end{figure} 


\begin{figure}[ht]
\centering
\includegraphics[trim = 40mm 92mm 40mm 92mm, clip,width=1\textwidth]{monthlyLEGPMali0105}
\caption{LEGP for each month, and corresponding solar resource. Model uses weather from Segou, Mali, and micro-grid demand profile with refrigerator base load.}
\label{monthlyLEGP}
\end{figure} 





\section{Discussion}
% goal is to integrate the information on weather and temporal patterns
% into a coherent story.  maybe discuss a thought process for choosing a
% sizing for a given demand?


\end{document}
